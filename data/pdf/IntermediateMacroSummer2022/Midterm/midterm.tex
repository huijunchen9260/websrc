\documentclass[14pt]{extarticle}

% \usepackage[style=authoryear,maxbibnames=9,maxcitenames=2,uniquelist=false,backend=biber,doi=false,url=false]{biblatex}
% \addbibresource{$BIB} % bibtex location
% \renewcommand*{\nameyeardelim}{\addcomma\space} % have comma in parencite
\usepackage{natbib}

\usepackage{xcolor}
\usepackage{amsmath}
\newcommand{\tuple}[1]{ \langle #1 \rangle }
%\usepackage{automata}
\usepackage{times}
\usepackage{ltablex}

%%%%%% Template
\usepackage{hyperref}
\hypersetup{colorlinks=true,allcolors=blue}

\usepackage{vmargin}
\setpapersize{USletter}
\setmarginsrb{1.0in}{1.0in}{1.0in}{0.6in}{0pt}{0pt}{0pt}{0.4in}

% HOW TO USE THE ABOVE:
%\setmarginsrb{leftmargin}{topmargin}{rightmargin}{bottommargin}{headheight}{headsep}{footheight}{footskip}
%\raggedbottom
% paragraphs indent & skip:
\parindent  0.3cm
\parskip    -0.01cm

\usepackage{tikz}
\usetikzlibrary{backgrounds}

% hyphenation:
% \hyphenpenalty=10000 % no hyphen
% \exhyphenpenalty=10000 % no hyphen
\sloppy

% notes-style paragraph spacing and indentation:
\usepackage{parskip}
\setlength{\parindent}{0cm}

% let derivations break across pages
\allowdisplaybreaks

\newcommand{\orange}[1]{\textcolor{orange}{#1}}
\newcommand{\blue}[1]{\textcolor{blue}{#1}}
\newcommand{\red}[1]{\textcolor{red}{#1}}
\newcommand{\freq}[1]{{\bf \sf F}(#1)}
\newcommand{\datafreq}[2]{{{\bf \sf F}_{#1}(#2)}}

\def\qqquad{\quad\qquad}
\def\qqqquad{\qquad\qquad}

%%%%%%%%%%%%%%%%%%%%%%%%%%%%%%%%%%%%%%%%%%%%%%%%%%%%%%%%%%%%%%%%%%%%%%%%%%%%%%%%
%%%%%%%%%%%%%%%%%%%%%%%%%%%%%%%%%%%%%%%%%%%%%%%%%%%%%%%%%%%%%%%%%%%%%%%%%%%%%%%%
\begin{document}

% \setcounter{section}{}
\centerline{\huge\bf ECON 4002.01 Midterm Exam}
\smallskip
\centerline{\LARGE Hui-Jun Chen}

\medskip

\section*{Instruction}
\label{sec:Instruction}
\addcontentsline{toc}{section}{Instruction}
Please submit your answer on Carmen Quiz ``Midterm Exam''.
All numerical answers are supposed to \red{\textbf{round to the second decimal point}},%
\footnote{round to the second decimal points means that if the third decimal point is a number between $ 0 $ to $ 4 $, then just get rid of the third decimal point. On the other hand, if the third decimal point is a number between $ 5 $ to $ 9 $, then round the second decimal point up by adding $ 1 $ to the second decimal point number. For example, if the answer you get is $ 0.534 $, then round it to $ 0.53 $. Yet, if the answer is $ 0.535 $, then round it up to $ 0.54 $.}
and all algebraic answers are supposed to


You \textbf{may} consult any note and textbook, but you \textbf{cannot} discuss with your classmate or any other person about the exam.

There will be one T/F choice question that worth 2 points in the Carmen Quiz ``Midterm Exam''. The T/F choice question is to confirm: ``\textbf{I affirm that I have not received or given any unauthorized help on this exam, and that all work is my own.}''

\section*{Question 1}
\label{sec:Question_1}
\addcontentsline{toc}{section}{Question 1}
Considering an one-period general equilibrium model similar to Example in Lecture 08, slide 11 and 12.
However, in this model economy, there are two differences:
\begin{enumerate}
    \item firm is choosing capital input, and consumers are endowed with $ 2 $ units of capital ($ K^{S} = 2 $).
    \item consumer's utility function is $ U(C, l) = \frac{C^{1-b}}{1-b} + \frac{l^{1-d}}{1-d} $
\end{enumerate}
The competitive equilibrium given
\textbf{\red{\underline{\quad \{ Q1 \} \quad}}} is a set of allocations
\textbf{\red{\underline{\quad \{ Q2 \} \quad}}} and prices
\textbf{\red{\underline{\quad \{ Q3 \} \quad}}}
such that
\begin{enumerate}
    \item
\end{enumerate}












% \bibliographystyle{$BIB_STYLE}
% \bibliography{$BIBFILE}

\end{document}

