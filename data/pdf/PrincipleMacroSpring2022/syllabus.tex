\documentclass[12pt]{article}

\usepackage[style=authoryear,maxbibnames=9,maxcitenames=2,uniquelist=false,backend=biber,doi=false,url=false]{biblatex}
\addbibresource{$BIB} % bibtex location
\renewcommand*{\nameyeardelim}{\addcomma\space} % have comma in parencite

\usepackage{xcolor}
 \usepackage{amsmath}
\newcommand{\tuple}[1]{ \langle #1 \rangle }
%\usepackage{automata}
\usepackage{times}
\usepackage{ltablex}

%%%%%% Template
\usepackage{hyperref}
\hypersetup{colorlinks=true,allcolors=blue}

\usepackage{vmargin}
\setpapersize{USletter}
\setmarginsrb{1.0in}{1.0in}{1.0in}{0.6in}{0pt}{0pt}{0pt}{0.4in}

% HOW TO USE THE ABOVE:
%\setmarginsrb{leftmargin}{topmargin}{rightmargin}{bottommargin}{headheight}{headsep}{footheight}{footskip}
%\raggedbottom
% paragraphs indent & skip:
\parindent  0.3cm
\parskip    -0.01cm

\usepackage{tikz}
\usetikzlibrary{backgrounds}

% hyphenation:
\sloppy

% notes-style paragraph spacing and indentation:
\usepackage{parskip}
\setlength{\parindent}{0cm}

% let derivations break across pages
\allowdisplaybreaks

\def\blue{\color{blue}}
\def\orange{\color{orange}}

\def\qqquad{\quad\qquad}
\def\qqqquad{\qquad\qquad}

%%%%%%%%%%%%%%%%%%%%%%%%%%%%%%%%%%%%%%%%%%%%%%%%%%%%%%%%%%%%%%%%%%%%%%%%%%%%%%%%
%%%%%%%%%%%%%%%%%%%%%%%%%%%%%%%%%%%%%%%%%%%%%%%%%%%%%%%%%%%%%%%%%%%%%%%%%%%%%%%%
\begin{document}

\centerline{\huge\bf Syllabus: ECON 2002.01-203 (19859)}
\medskip
\centerline{\LARGE \bf Principle of Macroeconomics}
\medskip
\centerline{\LARGE \bf Spring 2022}
\medskip
\centerline{\Large Instructor: Hui-Jun Chen}

\medskip

\section*{Course Overview}
\begin{itemize}
    \item Meeting Time: Monday, Wednesday, Friday, 8:00AM - 8:55AM
    \item Location: McPherson Lab 1015
    \item Class Dates: Jan 10, 2022-Apr 25, 2022
    \item Email address: \href{chen.9260@buckeyemail.osu.edu}{chen.9260@buckeyemail.osu.edu}.
    \item To get email reply, you \textbf{must} satisfy two conditions below, :
    \begin{enumerate}
        \item DO \textbf{NOT} SEND TO CARMEN EMAIL.
        \item Use \texttt{[E2002.01]} at the beginning of your subject title.
        \begin{itemize}
            \item example title: \texttt{[E2002.01] Question regarding Extra credit}
        \end{itemize}
    \end{enumerate}
    \item I will reply your email within \textit{2 business day}.
    \end{itemize}
    \item Office hour:
    \item Principles of macroeconomics will cover the following general topics: measures of national well-being, macroeconomic models, economic growth, monetary and fiscal policy.
\end{itemize}

\section*{Course learning outcomes}

\textbf{This course fulfills the GE Goals and Expected Learning Outcomes for Social Science: Organizations and Polities.}

\subsection*{Social Science Goal}

Students understand the systematic study of human behavior and cognition; the structure of human societies, cultures, and institutions; and the processes by which individuals, groups, and societies interact, communicate, and use human, natural, and economic resources.

\subsection*{Organizations and Polities Expected Learning Outcomes}
\begin{enumerate}
    \item Students understand the theories and methods of social scientific inquiry as they apply to the study of organizations and polities.
    \item Students understand the formation and durability of political, economic, and social organizing principles and their differences and similarities across contexts.
    \item Students comprehend and assess the nature and values of organizations and polities and their importance in social problem solving and policy making.
\end{enumerate}

Economics 2002.01 addresses the theories and methods of social scientific inquiry through discussion of supply and demand at the national level, and the measurement of national income and other macroeconomic measures, along with applications to current events.

Students will learn about the formation and durability of political, economic, and social organizing principles through discussions of the origin and structure of central banks as well as other international organizations, and fiscal and monetary policy. These topics will include discussion of various commonly accepted points of view.

Students will comprehend and assess the nature and values of organizations and polities and their importance in social problem solving and policy making through discussion of fiscal and monetary policy, business cycles and the Federal Reserve Bank, including its values and objectives.

\section*{Course materials}

Course website:

Please register as student on \href{https://www.core-econ.org/}{https://www.core-econ.org/}, and use the open source textbook \href{https://www.core-econ.org/the-economy/book/text/0-3-contents.html}{The Economy} as the major textbook for this course.
You can use the student resources in the core-econ website as the tool for evaluating your understanding about the course material.

\section*{Grades}

\section*{Grading Policy}

% This course puts a lot of weight on your performance during the semester, not just the exams. To perform well in the course, in addition to doing well in the exams, you need to do two more things: Homework assignments on core-econ and Quizzes on core-econ.
% Homework Assignment on core-econ: A good way to study the course material is through Homework Assignment on core-econ. Homework assignments are effort-based. For each week, you only need to upload 5 “exercise section” in The Economy textbook for the Unit taught in that week. Therefore, as long as you upload a file contains 5 exercises, you will get full credit for that homework assignment. Each chapter is worth 1 point. The answer of the homework assignment will be uploaded after the deadline of the corresponding homework assignment, so you can check whether your understanding on the material is correct. If I found you upload an empty file to cheat on this effort-based assignment, you’ll immediately fail this course.
% Quizzes on core-econ: For every chapter, core-econ provides quizzes for you to test your understanding about the chapter. You will answer the quizzes on Carmen. For each weekly quiz, you’ll answer 5 questions within one hour. Unlike Homework Assignment, there is no retake and there is no hint. So, you need some preparation before you take the quizzes. Each week’s quiz is worth 2 points. Take the quizzes very seriously because the exam questions will be very similar to midterm and final exam questions.

\subsection*{Extra credits}

Extra credits: At the end of the semester, I will have you do the Student Evaluation of Instruction (SEI). If I get $85\%$ response rate on SEI by the end of the semester, everyone will get 2 points of extra credits.

% Discussion board on Carmen: As the substitute for midterm, the weekly discussion board should play an important role in facilitating your understanding in math and economics. The youtube video series Essence of Calculus is an elegant and intuitive video to serve such purpose. Our textbook core-econ is known for using a lot of Calculus in its context. To facilitate discussion, the rule of the discussion board would work as follows: total points are 33 points, each week’s discussion are 3 points. In each week, you’ll need to watch one video in  Essence of Calculus that will be assigned on Carmen discussion board, and write (1) one review for the content of the video (1 points), and (2) two replies (1 point each) for other people’s review. The review should be more than 150 words, and replies should be more than 30 words. Any review or reply that does not match the word count standard only get a proportion of the points.

\subsection*{Curving}

If less than $40\%$ of the students get A- or above, I will add some points to everybody until $40\%$ of the students get A- or above, but I don’t expect this to occur.

\subsection*{Assignment / Quiz Due}

All assignments are due on the date specified on Carmen; late assignments are not accepted, unless you have formal excuse.

\subsection*{Grading scale}
\begin{itemize}
    \item 93–100: A
    \item 90–92.9: A-
    \item 87–89.9: B+
    \item 83–86.9: B
    \item 80–82.9: B-
    \item 77–79.9: C+
    \item 73–76.9: C
    \item 70 –72.9: C-
    \item 67 –69.9: D+
    \item 60 –66.9: D
    \item Below 60: E
\end{itemize}









\printbibliography

\end{document}

