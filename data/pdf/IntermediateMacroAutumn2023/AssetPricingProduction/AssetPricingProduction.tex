\documentclass[11pt,aspectratio=43,usenames,dvipsnames]{beamer}
\usepackage{fontawesome5}
\usepackage[utf8]{inputenc}
\usepackage{amsmath, amsfonts, amssymb, amsthm}
\usepackage[T1]{fontenc}
% mint: code chuck and syntax highlighting
%% outputdir should change according to pdf build directory
\usepackage[outputdir=build,cache=false]{minted}
\usepackage{lmodern}
\usepackage{xcolor}
\usepackage{setspace}
\usepackage{booktabs}
\usepackage{multirow}
\usepackage{graphicx}
\usepackage{tikz}
% \usetikzlibrary{decorations}
\usetikzlibrary{decorations.pathreplacing, intersections}
\usepackage{ulem}
\usepackage{hyperref}
\usepackage{booktabs}
\usepackage{babel}
\usepackage{makecell}
\usepackage[para,online,flushleft]{threeparttable}
\usepackage{pdfpages}
\usepackage{tcolorbox}
\usepackage{bm}
\usepackage{appendixnumberbeamer}
\usepackage{natbib}
\usepackage{caption}
\captionsetup[figure]{labelformat=empty}% redefines the caption setup of the figures environment in the beamer class.
\usetheme[compress]{Boadilla}
\usecolortheme{default}
\useoutertheme{miniframes}
\usefonttheme[onlymath]{serif}

\newcommand{\jump}[2]{\hyperlink{#1}{\beamerbutton{#2}}}
\newcommand{\orange}[1]{\textcolor{orange}{#1}}
\newcommand{\red}[1]{\textcolor{red}{#1}}
\newcommand{\blue}[1]{\textcolor{blue}{#1}}
\newcommand{\green}[1]{\textcolor{OliveGreen}{#1}}

\renewcommand{\square}{\scalebox{0.7}{$\blacksquare$ \hspace{0.5em}}}
\setbeamertemplate{itemize item}{\raisebox{0.1em}{\scalebox{0.7}{$\blacksquare$}}}
\setbeamertemplate{itemize subitem}[circle]
\setbeamertemplate{itemize subsubitem}{--}
\setbeamercolor{itemize item}{fg=black}
\setbeamercolor{itemize subitem}{fg=black}
\setbeamercolor{itemize subsubitem}{fg=black}
\setbeamercolor{item projected}{bg=darkgray,fg=white}
\definecolor{blue}{rgb}{0.2, 0.2, 0.7}
\setbeamercolor{alerted text}{fg=blue}
\setbeamertemplate{enumerate items}[circle]


\setbeamertemplate{headline}{}

%==========================================
\let\olditemize=\itemize
\let\endolditemize=\enditemize
\renewenvironment{itemize}{\olditemize \itemsep1em}{\endolditemize}
\let\oldenumerate=\enumerate
\let\endoldenumerate=\endenumerate
\renewenvironment{enumerate}{\oldenumerate \itemsep1em}{ \endoldenumerate}

\DeclareMathOperator*{\argmax}{\arg\!\max}
\DeclareMathOperator*{\E}{\mathbb{E}}
\DeclareMathOperator*{\var}{\rm Var}
\DeclareMathOperator*{\cov}{\rm Cov}

\theoremstyle{definition}
\newtheorem{assume}{Assumption}
\newtheorem{lem}{Lemma}
\newtheorem{proposition}{Proposition}
\newtheorem{thm}{Theorem}
\newtheorem{corol}{Corollary}

\AtBeginSection[]{
  \begin{frame}[noframenumbering]
  \vfill
  \centering
  \begin{beamercolorbox}[sep=8pt,center,shadow=true,rounded=true]{title}
    \usebeamerfont{title}\insertsection\par%
  \end{beamercolorbox}
  \vfill
  \end{frame}
}

\begin{document}
    \title[Asset Pricing (Production)]{Asset Pricing in Production Economy}
    \author[Hui-Jun Chen]{Hui-Jun Chen}
    \institute[OSU]{The Ohio State University}
    % \date{\today}
    \date{\today}
    \setbeamertemplate{navigation symbols}{}
    \setstretch{1.2}

%-------------------------------------------------------
{
%	\usebackgroundtemplate{\includegraphics[width=1\paperwidth]{../EveningSky_cropped_edit43_bright.jpg}}
    \begin{frame}
% \vspace{3em}
        \centering
%		{\footnotesize 	ECON 4002 Intermediate Macroeconomic Theory}
        \maketitle
% \vspace{-1.5em}
% \centering
% \includegraphics[width=0.55\linewidth]{Pictures/houses.jpeg}


    \end{frame}
}

% -------------------------------------------
\setbeamertemplate{headline}
{
\setbeamercolor{section in head/foot}{fg=black, bg=white}
\vskip1em \tiny \insertsectionnavigationhorizontal{1\paperwidth}{\hspace{0.50\paperwidth}}{}
}
%------------------------------------------

\begin{frame}{Overview}
\label{slide:Overview}
    \begin{center}
        How does share price comove with GDP?
    \end{center}
    \begin{itemize}
        \item We extend \cite{Lucas_1978} to production economy $ \Rightarrow  $ \textbf{firms}
        \item firms are \blue{active} player in macro: \textbf{investment v.s. GDP volatility}
        \begin{itemize}
            \item corporate finance: firm debt? capital investment?
            \item human resource: hiring / lay off employee?
            \item international economics: multi-nation enterprise? FDI?
        \end{itemize}
        \item To be able to reach some conclusion, we need simplification:
        \begin{itemize}
            \item similar setting as \cite{Lucas_1978}, representative HH \& firm
            \item firm pays dividend $ \Leftarrow  $ firm are \textbf{DRS}
            \item labor-only technology $ \Rightarrow  $ no other intertemporal asset other than share.
        \end{itemize}
    \end{itemize}
\end{frame}

\section[Firm]{Firm Problem}
\label{sec:Firm_Problem}

\begin{frame}{Dividend and Wage}
\label{slide:Dividend_and_Wage}
\begin{itemize}
    \item Production function: $ \displaystyle y = z n^{\alpha} $, where $ z $ is TFP shock, and $ \alpha \in (0, 1) $.
    \item Firm's profit maximization problem: $ \displaystyle \max_{n} z n^{\alpha} - wn $
    \begin{itemize}
        \item FOC: $ \displaystyle w = \alpha z n ^{\alpha - 1} $
    \end{itemize}
    \item Wage bill: $ \displaystyle w n = \alpha z n ^{\alpha} = \alpha y $
    \item Assume firm all profits as dividend, $ \displaystyle d = y - wn = (1-\alpha)y $
\end{itemize}

\end{frame}

\section[Household]{Household's Problem}
\label{sec:Household_s_Problem}

\begin{frame}{Household Problem}
\label{slide:Household_Problem}

Assume HH value leisure, and thus
%
\begin{align}
    V(s, z)
        & = \max_{c\ge 0, s'\ge 0, n \ge 0} \log c + \psi (1-n) + \beta \mathbb{E}_{z'|z} [V(s', z')]
    \\
    \text{s.t. } \quad
        & c + p s' \le (d + p) s + wn
\end{align}
%
We know in equilibrium / steady state, three markets need to clear:
\begin{enumerate}
    \item find $ w $ such that labor demand $ = $ labor supply
    \item find $ p $ such that $ s = 1 $
    \item by Walras' law, goods market clear, implying $ c = y $.
\end{enumerate}

\end{frame}

\begin{frame}{Solve Household Problem}
\label{slide:Solve_Household_Problem}
    Using the same solution technique,
    %
    \begin{align}
        V(s, z)
            & = \max_{s', c, n} \log c + \psi (1-n) + \beta \mathbb{E}_{z'|z}[\log c' + \psi (1-n')]
        \\
            & \quad + \beta^{2} \mathbb{E}_{z'|z} [V(s'', z'')]
        \\
        \text{subject to } \quad
            & c + ps' \le (d + p)s + wn
        \\
            & c' + p's'' \le (d' + p') s' + w' n'
    \end{align}
    %
    Replace $ c $ and $ c' $ and get
    %
    \begin{align}
        V(s, z)
            & = \max_{s', n} \log((d+p)s + wn - p s') + \psi (1-n)
        \\
            & \quad + \beta \mathbb{E}_{z'|z} [\log((d'+p')s' + w'n' - p's'') + \psi (1 - n')]
        \\
            & \quad + \beta^{2} \mathbb{E}_{z'|z} [V(s'', z'')]
    \end{align}
    %
\end{frame}

\begin{frame}{First Order Condition}
\label{slide:First_Order_Condition}
    %
    \begin{align}
        V(s, z)
            & = \max_{s', n} \log((d+p)s + wn - p s') + \psi (1-n)
        \\
            & \quad + \beta \mathbb{E}_{z'|z} [\log((d'+p')s' + w'n' - p's'') + \psi (1 - n')]
        \\
            & \quad + \beta^{2} \mathbb{E}_{z'|z} [V(s'', z'')]
    \end{align}
    %
    FOC:
    %
    \begin{align}
        [n]: \quad
            & \frac{w}{c} = \psi
        \\
        [s']: \quad
            & \frac{1}{c} \cdot p = \beta \mathbb{E}_{z'|z} \left[
                \frac{1}{c'} \cdot (d' + p')
            \right]
    \end{align}
    %
\end{frame}

\section[Eqm]{Equilibrium Outcome}
\label{sec:Equilibrium_Outcome}

\begin{frame}{Optimality Conditions}
\label{slide:Optimality_Conditions}
    \begin{align}
        \label{eq:NFOC}
        [n]: \quad
            & \frac{w}{c} = \psi \Rightarrow w = \psi c
        \\
        \label{eq:SprimeFOC}
        [s']: \quad
            & \frac{1}{c} \cdot p = \beta \mathbb{E}_{z'|z} \left[
                \frac{1}{c'} \cdot (d' + p')
            \right]
        \\
        \label{eq:NDFOC}
        [\text{Firm}]: \quad
            & w = \alpha z n^{\alpha-1}
    \end{align}
    $ w = w $, \eqref{eq:NFOC} equals to \eqref{eq:NDFOC}, and $ c = y $ yields
    %
    \begin{align}
        \psi y = \alpha \frac{y}{n}
            & \Rightarrow n = \frac{\alpha}{\psi} \Rightarrow y = z n^{\alpha} = z \left(
                \frac{\alpha}{\psi}
            \right)^{\alpha}
        \\
            & \Rightarrow w = \alpha z n ^{\alpha-1} = \alpha z \left(
                \frac{\alpha}{\psi}
            \right)^{\alpha-1}
        \\
            & \Rightarrow d = (1-\alpha) y = (1-\alpha) z \left(
                \frac{\alpha}{\psi}
            \right)^{\alpha}
    \end{align}
    %

\end{frame}

\begin{frame}{Share Euler Equation}
\label{slide:Share_Euler_Equation}
    Focus on \eqref{eq:SprimeFOC}, we can use $ c' = y' $ as well as $ d' = (1-\alpha) y' $ to simplify:
    %
    \begin{align}
        \frac{p}{y}
            & = \beta \mathbb{E}_{z'|z} \left[
                \frac{p'}{y'} + \frac{(1-\alpha)y'}{y'}
            \right]
        \\
            & = \beta (1-\alpha) + \beta \mathbb{E}_{z'|z} \left[
                \frac{p'}{y'}
            \right]
    \end{align}
    %
    Somehow you got a prophecy from the spirit and his/her voice tells you to guess $ \displaystyle \frac{p}{y} \equiv \Lambda $, a constant over time regardless of TFP shock. Is that true?
    %
    \begin{align}
        \Lambda
            & = \beta (1-\alpha) + \beta \mathbb{E}_{z'|z} \left[
                \Lambda
            \right] = \beta(1-\alpha) + \beta \Lambda
        \\
        \Lambda
            & = \frac{\beta(1-\alpha)}{1-\beta}
    \end{align}
    %
    \begin{center}
        It true \faAngellist \faAngellist \faAngellist
    \end{center}
\end{frame}

\begin{frame}{Intepretation}
\label{slide:Intepretation}
    Stock price to GDP ratio, $ \frac{p}{y} $, is constant over time, which implies
    \begin{enumerate}
        \item stock price is procyclical: \faArrowCircleUp \hspace{1px} and \faArrowCircleDown \hspace{1px} with TFP $ z $,
        \item the percentage std of stock price matches percentage std of dividend,
        \item stock is risky: $ \displaystyle p = \frac{\beta(1-\alpha)}{1-\beta} y $ $ \Rightarrow  $ requires $ (+) $ risk premium
        \begin{itemize}
            \item $ \displaystyle e(z, z') = \frac{d'+p'}{p} = \frac{(1-\alpha)y' + \Lambda y'}{\Lambda y} = \frac{\frac{1-\alpha}{1-\beta}y'}{ \frac{\beta(1-\alpha)}{1-\beta} y} = \frac{1}{\beta} \frac{y'}{y}$
            \item SDF $ \displaystyle = \frac{\beta u'(c')}{u'(c)} = \beta \frac{y}{y'} $
            \item Risk premium $ \displaystyle = \frac{\mathbb{E}_{t}[e(z, z') - R_{t}]}{R_{t}} = - cov_{t} \left[
                SDF, e(z, z')
            \right] > 0$
        \end{itemize}
    \end{enumerate}
    The very times firm shares pay high is when your consumption is low!
\end{frame}

\section{Appendix}
\label{sec:Appendix}

\appendix
% -------------------------------------------
\setbeamertemplate{headline}
{
\setbeamercolor{section in head/foot}{fg=black, bg=white}
\vskip1em \tiny \insertsectionnavigationhorizontal{1\paperwidth}{\hspace{0.50\paperwidth}}{}
}
%------------------------------------------
% \begin{frame}\frametitle{}
% \begin{columns}
% \label{Appendix}
% \column{1\linewidth}
% \centering
% {\Large \alert{Appendix}}
% \end{columns}
% \end{frame}
%------------------------------------------
\begin{frame}[allowframebreaks]{References}
\footnotesize
\bibliographystyle{$BIB_STYLE}
\bibliography{$BIBFILE}
\end{frame}

\end{document}
