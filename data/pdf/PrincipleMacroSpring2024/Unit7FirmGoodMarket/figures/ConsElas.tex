\documentclass[tikz]{standalone}

\usepackage{tikz}
\usetikzlibrary{decorations}
\usetikzlibrary{decorations.pathreplacing, intersections}
\usepackage{pgfplots}
\usetikzlibrary{calc,positioning}
\pgfplotsset{compat=newest, scale only axis, width = 10cm}

% Create fake \onslide and other commands for standalone picture
\usepackage{xparse}
\NewDocumentCommand{\onslide}{s t+ d<>}{}
\NewDocumentCommand{\only}{d<>}{}
\NewDocumentCommand{\uncover}{d<>}{}
\NewDocumentCommand{\visible}{d<>}{}
\NewDocumentCommand{\invisible}{d<>}{}

\begin{document}

\begin{tikzpicture}

\begin{axis}[
    xmin = 0,
    xmax = 10,
    ymin = 0,
    ymax = 3,
    xlabel = {$Q$},
    ylabel = {$P$},
    sciclean/.style={axis lines=left,
        axis x line shift=0.5em,
        axis y line shift=0.5em,
        axis line style={-,very thin},
        axis background/.style={draw,ultra thin,gray},
        tick align=outside,
        major tick length=2pt},
    xtick distance=1,
    ytick distance=0.5,
    title = {Constant Elasticity of Demand},
    domain = 0.4:10,
    sciclean]

    \addplot[thick, color = blue, samples = 1000] {x^(-(1/0.8))};
    \addlegendentry{$Q = P^{-0.8}$}

\end{axis}

    % We could control parts of figure only shown in beamer or vice versa.
    % \ifstandalone
    %     \node[below=1cm of mid] {Only Shown in Standalone Figure};
    % \else
    %     \node[below=1cm of mid] {Only Shown in Beamer};
    % \fi
\end{tikzpicture}

\end{document}
