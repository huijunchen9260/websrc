\documentclass[11pt,aspectratio=43,usenames,dvipsnames]{beamer}
\usepackage[utf8]{inputenc}
\usepackage{amsmath, amsfonts, amssymb, amsthm}
\usepackage[T1]{fontenc}
% mint: code chuck and syntax highlighting
%% outputdir should change according to pdf build directory
\usepackage[outputdir=build,cache=false]{minted}
\usepackage{lmodern}
\usepackage{xcolor}
\usepackage{setspace}
\usepackage{booktabs}
\usepackage{multirow}
\usepackage{graphicx}

\usepackage[mode=tex]{standalone}
\usepackage{tikz}
\usetikzlibrary{decorations}
\usetikzlibrary{decorations.pathreplacing, intersections}
\usepackage{pgfplots}
\usetikzlibrary{calc,positioning}
\usepgfplotslibrary{fillbetween}
\pgfplotsset{compat=newest, scale only axis, width = 10cm}

\newcommand{\Getxycoords}[3]{%
    \pgfplotsextra{%
        % using `\pgfplotspointgetcoordinates' stores the (axis)
        % coordinates in `data point' which then can be called by
        % `\pgfkeysvalueof' or `\pgfkeysgetvalue'
        \pgfplotspointgetcoordinates{(#1)}%
        % `\global' (a TeX macro and not a TikZ/PGFPlots one) allows to
        % store the values globally
         \global\pgfkeysgetvalue{/data point/x}{#2}%
         \global\pgfkeysgetvalue{/data point/y}{#3}%
     }%
}

% % Create fake \onslide and other commands for standalone picture
% \usepackage{xparse}
% \RenewDocumentCommand{\onslide}{s t+ d<>}{}
% \RenewDocumentCommand{\only}{d<>}{}
% \RenewDocumentCommand{\uncover}{d<>}{}
% \RenewDocumentCommand{\visible}{d<>}{}
% \RenewDocumentCommand{\invisible}{d<>}{}

% \tikzset{
%     onslide/.code args={<#1>#2}{% http://tex.stackexchange.com/a/6155/16595
%         \only<#1>{\pgfkeysalso{#2}}
%     },
%     alt/.code args={<#1>#2#3}{%
%         \alt<#1>{\pgfkeysalso{#2}}{\pgfkeysalso{#3}} % \pgfkeysalso doesn't change the path
%     },
% }

\usepackage{ulem}
\usepackage{hyperref}
\usepackage{booktabs}
\usepackage{babel}
\usepackage{makecell}
\usepackage[para,online,flushleft]{threeparttable}
\usepackage{pdfpages}
\usepackage{tcolorbox}
\usepackage{bm}
\usepackage{appendixnumberbeamer}
\usepackage{natbib}
\usepackage{caption}
\captionsetup[figure]{labelformat=empty}% redefines the caption setup of the figures environment in the beamer class.
\usetheme[compress]{Boadilla}
\usecolortheme{default}
\useoutertheme{miniframes}
\usefonttheme[onlymath]{serif}

\newcommand{\jump}[2]{\hyperlink{#1}{\beamerbutton{#2}}}
\newcommand{\orange}[1]{\textcolor{orange}{#1}}
\newcommand{\red}[1]{\textcolor{red}{#1}}
\newcommand{\blue}[1]{\textcolor{blue}{#1}}
\newcommand{\green}[1]{\textcolor{OliveGreen}{#1}}

\renewcommand{\square}{\scalebox{0.7}{$\blacksquare$ \hspace{0.5em}}}
\setbeamertemplate{itemize item}{\raisebox{0.1em}{\scalebox{0.7}{$\blacksquare$}}}
\setbeamertemplate{itemize subitem}[circle]
\setbeamertemplate{itemize subsubitem}{--}
\setbeamercolor{itemize item}{fg=black}
\setbeamercolor{itemize subitem}{fg=black}
\setbeamercolor{itemize subsubitem}{fg=black}
\setbeamercolor{item projected}{bg=darkgray,fg=white}
\definecolor{blue}{rgb}{0.2, 0.2, 0.7}
\setbeamercolor{alerted text}{fg=blue}
\setbeamertemplate{enumerate items}[circle]


\setbeamertemplate{headline}{}

%==========================================
\let\olditemize=\itemize
\let\endolditemize=\enditemize
\renewenvironment{itemize}{\olditemize \itemsep1em}{\endolditemize}
\let\oldenumerate=\enumerate
\let\endoldenumerate=\endenumerate
\renewenvironment{enumerate}{\oldenumerate \itemsep1em}{ \endoldenumerate}

\DeclareMathOperator*{\argmax}{\arg\!\max}
\DeclareMathOperator*{\E}{\mathbb{E}}
\DeclareMathOperator*{\var}{\rm Var}
\DeclareMathOperator*{\cov}{\rm Cov}

\theoremstyle{definition}
\newtheorem{assume}{Assumption}
\newtheorem{lem}{Lemma}
\newtheorem{proposition}{Proposition}
\newtheorem{thm}{Theorem}
\newtheorem{corol}{Corollary}

\AtBeginSection[]{
  \begin{frame}[noframenumbering]
  \vfill
  \centering
  \begin{beamercolorbox}[sep=8pt,center,shadow=true,rounded=true]{title}
    \usebeamerfont{title}\insertsection\par%
  \end{beamercolorbox}
  \vfill
  \end{frame}
}

\begin{document}
    \title[Unit 7]{Unit 7: \\ The Firm, Demand Elasticity \\ and Market Competition}
    \author[Hui-Jun Chen]{Hui-Jun Chen}
    \institute[OSU]{The Ohio State University}
    \date{\today}
    \setbeamertemplate{navigation symbols}{}
    \setstretch{1.2}

%-------------------------------------------------------
{
%	\usebackgroundtemplate{\includegraphics[width=1\paperwidth]{../EveningSky_cropped_edit43_bright.jpg}}
    \begin{frame}
% \vspace{3em}
        \centering
%		{\footnotesize 	ECON 4002 Intermediate Macroeconomic Theory}
        \maketitle
% \vspace{-1.5em}
% \centering
% \includegraphics[width=0.55\linewidth]{Pictures/houses.jpeg}


    \end{frame}
}

% -------------------------------------------
\setbeamertemplate{headline}
{
\setbeamercolor{section in head/foot}{fg=black, bg=white}
\vskip1em \tiny \insertsectionnavigationhorizontal{1\paperwidth}{\hspace{0.50\paperwidth}}{}
}
%------------------------------------------

\section[Intro]{Introduction}
\label{sec:Introduction}

\begin{frame}{Introduction}
\label{slide:Introduction}
    \begin{center}
        \textit{How do the firm and consumers interacts?}
    \end{center}
    \begin{itemize}
        \item We have been neglecting \alert{revenue} for the past units. What's its deal?
        \item To answer this question, we need to see how consumers look like \alert{from firm's perspective}
        \item Firm doesn't see consumer as individuals; what they see is \textbf{demand}
        \begin{itemize}
            \item How sensitive the demand is to the \alert{prices}? (price elasticity)
            \item Can I produce enough to satisfy all demand? (returns to scale)
            \item Can I \alert{alter} the demand / set prices? (market power)
            \item How I benefit from production and trade? (producer/consumer surplus)
        \end{itemize}
    \end{itemize}
\end{frame}

\section[Elasticity]{Elasticity \& Price Elasticity of Demand}
\label{sec:Elasticity____Price_Elasticity_of_Demand}


\begin{frame}{Elasticity in general}
\label{slide:Elasticity_in_general}

\begin{definition}[The x-elasticity of y]
 The x-elasticity of y measures the fractional response of y to a fraction change in x
\end{definition}

\begin{itemize}
    \item Elasticity is the measure of the \textbf{sensitivity} of one variable to another.
    \item A \textbf{highly elastic} variable will \textbf{respond more dramatically} to changes in the variable it is dependent on
    % \item In the def above, $ y $ is the variable that being affected, and $ x $ is the variable that $ y $ depends on.
    \item The formula for elasticity is
        %
        \begin{align*}
         \frac{\text{growth rate of } y}{\text{growth rate of } x} \quad \text{or} \quad \frac{(y_{2} - y_{1}) / y_{1}}{(x_{2} - x_{1}) / x_{1}}
         &
         \quad
         \text{if discrete,}
         \\
         \frac{\partial y / y}{ \partial x / x}
        &
         \quad
        \text{if continuous.}
        \end{align*}
        %
\end{itemize}

\end{frame}

\begin{frame}{Price Elasticity of Demand}
\label{slide:Price_Elasticity_of_Demand}
    Following the definition of elasticity, the specification on \alert{how sensitive quantity demanded ($y$) is to the price ($x$)} is $ \displaystyle \epsilon = \frac{(Q_{2} - Q_{1})/Q_{1}}{(P_{2} - P_{1}) / P_{1}} $
    \begin{columns}
        \begin{column}{0.5\textwidth}
            Assume change in price is $ 1 $,
            \begin{itemize}
                \item $ |\epsilon_{A}| = \left|\frac{(20 - 20.0125 )/20}{(6400-6399)/6400}\right| = 4$
                \begin{itemize}
                    \item $6399 = -80 \times 20.0125 + 8000$
                \end{itemize}
                \item $ |\epsilon_{B}| = \left|\frac{(50 - 50.0125 )/20}{(4000-3999)/4000}\right| = 1$
                \begin{itemize}
                    \item $3999 = -80 \times 50.0125 + 8000$
                \end{itemize}
                \item $ |\epsilon_{C}| = \left|\frac{(70 - 70.0125 )/20}{(2400-2399)/6400}\right| = 0.43$
                \begin{itemize}
                    \item $2399 = -80 \times 70.0125 + 8000$
                \end{itemize}

            \end{itemize}

        \end{column}
        \begin{column}{0.5\textwidth}
            \begin{figure}
                \centering
                \includestandalone[width=\linewidth]{./figures/PElasQ}
            \end{figure}
        \end{column}
    \end{columns}


\end{frame}


\begin{frame}{Elasticity and Slope: History Story}
\label{slide:Elasticity_and_Slope}
    \begin{center}
        \textit{Is price determining quantity demanded, or the other way around?}
    \end{center}
    \begin{columns}
        \begin{column}{0.5\textwidth}
            \only<1>{
            \begin{itemize}
                \item Mathematically speaking, above question is asking $ Q(P) $ or $ P(Q) $
                \item Graphically speaking, $ QP $ plane is saying $ P(Q) $, i.e., \textit{quantity demanded determines prices} \ldots match experience?
                \item What is the ``slope'' in the axis-swapped figure? Roughly elasticity?
            \end{itemize}
            }
            \only<2>{
            \begin{itemize}
                \item Slope are $ -\frac{1}{80} $, the inverse of the slope before, and constant over A, B, and C.
                \item Turns out the original formulation is this figure, i.e., \alert{price determines the quantity demanded}, and we can measure the \textbf{absolute change} of demand using slope.
            \end{itemize}
            }
            \only<3>{
            \begin{itemize}
                \item However, some Economists wants to compare \alert{growth in quantity demanded} when \alert{prices are in a certain region} over a long time series, which motivates them to \alert{swap the axis}.
                \item But they still want to see how \alert{quantity changes with the price}!
            \end{itemize}
            }
            \only<4>{
            \begin{itemize}
                \item Eventually, as time goes by, we separate the definition as:
                \item Slope measures the \textbf{absolute/average changes}
                \item Elasticity measure the \textbf{relative/percentage/marginal changes}
                \item Thus, a straight line has \alert{constant slope} but \alert{different elasticity}
            \end{itemize}
            }
        \end{column}
        \begin{column}{0.5\textwidth}
            \only<1-2>{
            \begin{figure}
                \centering
                \includestandalone[width=\linewidth]{./figures/SlopeElas}
            \end{figure}
            }
            \only<3-4>{
            \begin{figure}
                \centering
                \includestandalone[width=\linewidth]{./figures/PElasQ}
            \end{figure}
            }
        \end{column}
    \end{columns}

\end{frame}

\begin{frame}{Constant Elasticity of Demand}
\label{slide:Constant_Elasticity_of_Demand}
    Since a straight line doesn't provide constant elasticity, what's the shape of demand function has constant elasticity?
    \begin{columns}
        \begin{column}{0.5\textwidth}
            \begin{itemize}
                \item Recall $ \epsilon = \frac{P}{Q(P)} \frac{\partial Q(P)}{\partial P} $, and $ Q(P) $ is a function of $ P $.
                \item $ Q(P) $ needs to have some \alert{power} so that after differentiation, all the $ P $'s cancels out and left only constant.
                \item If $ Q(P) = P^{-0.8} $, then $ \epsilon = \frac{P}{P^{-0.8}} \frac{\partial (P^{-0.8})}{\partial P} = \frac{P}{P^{-0.8}} \times  (-0.8) P^{-1.8} = -0.8 $
            \end{itemize}

        \end{column}
        \begin{column}{0.5\textwidth}
            \begin{figure}
                \centering
                \includestandalone[width=\linewidth]{./figures/ConsElas}
            \end{figure}
        \end{column}
    \end{columns}

\end{frame}

\section[Production]{Production: Key Concept}
\label{sec:Production__Key_Concept}

\begin{frame}{Economies of Scale / Return to Scale}
\label{slide:Economies_of_Scale___Return_to_Scale}
\begin{itemize}
    \item \textbf{Return to scale}: how output will change when inputs increase
    \item \textbf{Constant return to scale} (CRS): $ x z F( K, N^{d} ) = z F( xK, xN^{d} ) $
    \begin{itemize}
        \item output increase \alert{proportionally} with inputs
        \item small firms are \alert{as efficient as} large firms
    \end{itemize}
    \item \textbf{Increasing return to scale} (IRS): $ x z F( K, N^{d} ) > z F( xK, xN^{d} ) $
    \begin{itemize}
        \item output increase \alert{more than proportionally} with inputs
        \item small firms are \alert{less efficient than} large firms
    \end{itemize}
    \item \textbf{Decreasing return to scale} (DRS): $ x z F( K, N^{d} ) < z F( xK, xN^{d} ) $
    \begin{itemize}
        \item output increase \alert{less than proportionally} with inputs
        \item small firms are \alert{more efficient than} large firms
    \end{itemize}
\end{itemize}
\end{frame}

\begin{frame}{Economies of Scale: Example}
\label{slide:Economies_of_Scale__Example}
    \begin{itemize}
        \item IRS $ \rightarrow  $ Economies of scale, and DRS $ \rightarrow  $ Diseconomies of scale
        \item Economies of scale includes:
        \begin{enumerate}
            \item Cost advantages – Large firms can purchase inputs on more favourable terms, because they have greater bargaining power when negotiating with suppliers.
            \item Demand advantages - Network effects (value of output rises with number of users e.g. software application)
        \end{enumerate}
        \item However, large firms can also suffer from diseconomies of scale
        \begin{itemize}
            \item e.g. additional layers of bureaucracy due to too many employees.
        \end{itemize}
    \end{itemize}
\end{frame}

\begin{frame}{Cost Function}
\label{slide:Cost_Function}
    Cost functions show how production costs vary with quantity produced.
    \begin{columns}
        \begin{column}{0.45\textwidth}
            \begin{itemize}
                \only<1> { \item $ AC(Q) \equiv \frac{C(Q)}{Q} $: average cost }
                \only<1> { \item $ MC(Q) \equiv \frac{\partial C(Q)}{\partial Q} $: marginal cost }
                \only<1> { \item $ F $: fixed cost }
                \item $ c_{0} $: lowest point on $ AC(Q) $
                \only<1> { \item Why $ AC(Q) $ and $ MC(Q) $ intersect at the lowest point? }
                \only<2> { \item $ MC(Q) $ always increase as $ Q $ increases }
                \only<2> { \item If $ AC(Q) > \green{(<)} MC(Q) $: the relative increment in cost function, i.e., \alert{marginal cost}, is \textbf{smaller \green{(larger)}} than the \alert{increment of $ 1 $ unit $ Q $} (denominator), and thus $ AC(Q) \downarrow \green{(\uparrow )}  $}
            \end{itemize}
        \end{column}
        \begin{column}{0.6\textwidth}
            \begin{figure}
                \centering
                \includestandalone[width=\textwidth]{./figures/CostFunc}
            \end{figure}
        \end{column}
    \end{columns}
\end{frame}

\section[Profit]{Profit Maximization}
\label{sec:Profit_Maximization}

\begin{frame}{If Price is a Function of Quantity}
\label{slide:If_Price_is_a_Function_of_Quantity}
    \begin{columns}
        \begin{column}{0.4\textwidth}
            \begin{itemize}
                \only<1>{
                \item Assume the firm is \alert{monopoly}: price being affected by quantity decision
                \item Profit max: $ \pi = R(Q) - C(Q) $
                \item $ \frac{\partial R(Q)}{\partial Q}  \Rightarrow MR(Q) $
                \item $ \frac{\partial C(Q)}{\partial Q}  \Rightarrow MC(Q) $
                }
                \only<2>{
                \item FOC: $MR - MC = 0 \Rightarrow MR = MC$
                \item Intersect at $ E $, which determines optimal $ Q = 30.97 $
                \item As firm produce at $ Q = 30.97 $, the market price $ P = 90 - 30.97 = 59.03 $.
                }
                \only<3>{
                \item Another way to maximize profit is by \alert{isoprofit curve} and demand itself.
                \item $ MC $ is also the \alert{individual supply curve}
                }
            \end{itemize}

        \end{column}
        \begin{column}{0.6\textwidth}
            \begin{figure}
                \centering
                \includestandalone[width=\textwidth]{./figures/MRMC}
            \end{figure}
        \end{column}
    \end{columns}


\end{frame}

\begin{frame}{If Price is NOT a Function of Quantity (Residual Demand)}
\label{slide:If_Price_is_NOT_a_Function_of_Quantity}
    \begin{columns}
        \begin{column}{0.4\textwidth}
            \begin{itemize}
                \only<1>{
                \item Assume the firm is in \alert{perfect competition}: \alert{infinite number} of firms and each \alert{taken prices as given} (no market power)
                \item This tiny firm thinks it is facing a \alert{horizontal demand curve}, which means that he \alert{cannot affect prices with quantity produced}
                }
                \only<2>{
                \item The demand that \textbf{firm is perceiving} is called \alert{residual demand}
                \item $ P = 50 \Rightarrow R(Q) = 50Q \Rightarrow MR = 50 $
                \item $ P = MR = MC \Rightarrow 1.8 Q^{0.8} = 50 \Rightarrow Q = ( \frac{50}{1.8} )^{ \frac{1}{0.8}} \approx 63.77 $
                }
            \end{itemize}

        \end{column}
        \begin{column}{0.6\textwidth}
            \begin{figure}
                \centering
                \includestandalone[width=\textwidth]{./figures/MRMC_PC}
            \end{figure}

        \end{column}
    \end{columns}

\end{frame}

\section[Surplus]{Gains from Trade}
\label{sec:Gains_from_Trade}

\begin{frame}{Individual Supply Aggregates into Aggregate Supply}
\label{slide:Individual_Supply_Aggregation_into_Aggregate_Supply}
    \begin{columns}
        \begin{column}{0.45\textwidth}
            \begin{itemize}
                \item \alert{Tiny} firm 1 is facing residual demand $ P_{1} = 50 $, and thus he wants to produce at $ Q \approx 63.77 $
                \item \alert{Tiny} firm 2: $ P_{2} = 40$, produce $Q = ( \frac{40}{1.8} )^{ \frac{1}{0.8}} \approx 48.25$
                \item Same applies to firm 3 and 4, and thus even all firms have the same cost function, \alert{each point at supply curve represent each firm}.
            \end{itemize}

        \end{column}
        \begin{column}{0.6\textwidth}
            \begin{figure}
                \centering
                \includestandalone[width=\textwidth]{./figures/MRMC_PCFirm}
            \end{figure}
        \end{column}
    \end{columns}

\end{frame}

\begin{frame}{Individual Demand Aggregates into Aggregate Demand}
\label{slide:Individual_Demand_Aggregates_into_Aggregate_Demand}
    \begin{columns}
        \begin{column}{0.45\textwidth}
            \begin{itemize}
                \item \alert{Tiny} \alert{consumer} 1 is facing residual supply $ P_{1} = 50 $, and thus he wants to buy at $ Q = 40$
                \item \alert{Tiny} consumer 2: $ P_{2} = 40$, buy at $Q = 50$
                \item Same applies to consumer 3 and 4, and thus even all consumers have the same demand function, \alert{each point at demand curve represent each consumer}.
            \end{itemize}

        \end{column}
        \begin{column}{0.6\textwidth}
            \begin{figure}
                \centering
                \includestandalone[width=\textwidth]{./figures/MRMC_PCConsumer}
            \end{figure}
        \end{column}
    \end{columns}

\end{frame}

\begin{frame}{Consumer and Producer Surplus}
\label{slide:Consumer_and_Producer_Surplus}

    \begin{columns}
        \begin{column}{0.3\textwidth}
            \begin{itemize}
                \only<1>{
                \item Consumer $A$ is willing to pay $P_{A} = 80$
                \item Firm $ A' $ is will to produce at cost $ P_{A'} = 1.8 \times 10^{0.8} \approx 11.36 $
                \item Both pay $ P^{*} = 40.7 $:
                }
                \only<2>{
                \item Consumer $B$ is willing to pay $P_{B} = 70$
                \item Firm $ B' $ is will to produce at cost $ P_{B'} = 1.8 \times 20^{0.8} \approx 19.77 $
                \item Both pay $ P^{*} = 40.7 $:
                }
                \only<3>{
                \item Consumer $C$ is willing to pay $P_{C} = 60$
                \item Firm $ C' $ is will to produce at cost $ P_{C'} = 1.8 \times 30^{0.8} \approx 27.35$
                \item Both pay $ P^{*} = 40.7 $:
                }
                \only<4>{
                \item Consumer $D$ is willing to pay $P_{D} = 50$
                \item Firm $ D' $ is will to produce at cost $ P_{D'} = 1.8 \times 40^{0.8} \approx 34.43$
                \item Both pay $ P^{*} = 40.7 $:
                }
            \end{itemize}
        \end{column}
        \begin{column}{0.7\textwidth}
        \begin{figure}
            \centering
            \includestandalone[width=\textwidth]{./figures/MRMC_PCMarket}
        \end{figure}
        \end{column}
    \end{columns}

\end{frame}

\begin{frame}{Market Power}
\label{slide:Market_Power}
    \begin{center}
        \textit{What if firm is monopoly?}
    \end{center}
    \begin{columns}
        \begin{column}{0.4\textwidth}
            \begin{itemize}
                \item $ MR = MC $ determines $ Q^{*} = 31 $, and vertical upward to Demand at $ M_{2} $ to get $ P^{*} = 59 $.
                \item Deadweight Loss created: $ \triangle EM_{1}M_{2} $
                \item CS $ \downarrow$ $\because$ DWL \& PS
                \item PS $ \uparrow  $: gave up $ \triangle EM_{1}M_{3} $, but gain part of CS
            \end{itemize}

        \end{column}
        \begin{column}{0.6\textwidth}
        \begin{figure}
            \centering
            \includestandalone[width=\textwidth]{./figures/MRMC_MonopolyMarket}
        \end{figure}
        \end{column}
    \end{columns}

\end{frame}

\begin{frame}{Markup / Price Spread}
\label{slide:Markup___Price_Spread}
    \begin{itemize}
        \item Markup $ = \frac{P - MC}{P} $ is a measure of market power
        \item Graphically represented by $ \overline{M_{1}M_{2}} $, is inversely proportional to price elasticity of demand.
    \end{itemize}

    \begin{columns}
        \begin{column}{0.5\textwidth}
            \begin{figure}
                \centering
                \includestandalone[width=\textwidth]{./figures/MRMC_MonopolyMarket}
            \end{figure}
        \end{column}
        \begin{column}{0.5\textwidth}
            \begin{figure}
                \centering
                \includestandalone[width=\textwidth]{./figures/MRMC_Markup_LowPElas}
            \end{figure}
        \end{column}
    \end{columns}

\end{frame}

\begin{frame}[allowframebreaks]{Price Elasticity and Market Power}
\label{slide:Price_Elasticity_and_Market_Power}
\begin{itemize}
    \item A firm’s profit margin depends on the \alert{elasticity of demand}, which is determined by competition:
    \begin{itemize}
        \item Demand is relatively \textbf{inelastic} if there are \textbf{few close substitutes}
        \item Firms with market power have enough bargaining power to \alert{set prices} without losing customers to competitors
    \end{itemize}
    \item Competition policy (limits on market power) can be beneficial to consumers when firms collude to keep prices high.
    \framebreak
    \item Example of market power: A firm selling \alert{specialized} products.
    \begin{itemize}
        \item They face \alert{little competition} and hence have \alert{inelastic demand}.
        \item They can set price above marginal cost without losing customers, thus earning \textbf{monopoly rents}.
        \item This is a form of market failure because there is deadweight loss.
    \end{itemize}
    \item A natural monopoly arises when one firm can \alert{produce at lower average costs} ($+$) than two or more firms e.g. utilities.
    \item Instead of encouraging competition, policymakers may put price controls or make these firms publicly owned.
    \framebreak
    \item Firms can increase their market power by:
    \begin{enumerate}
        \item Innovating – Technological innovation can allow firms to differentiate their products from competitors’ e.g. hybrid cars.
        \begin{itemize}
            \item Firms that invent a completely new product may prevent competition altogether through patents or copyright laws.
        \end{itemize}

        \item Advertising – Firms can attract consumers away from competing products and create brand loyalty. Advertising can be more effective than discounts in increasing demand for a brand.
    \end{enumerate}
    \item Both of these tactics can shift the firm’s demand curve.
\end{itemize}
\end{frame}


%\section{Appendix}
%\label{sec:Appendix}

%\appendix
%% -------------------------------------------
%\setbeamertemplate{headline}
%{
%\setbeamercolor{section in head/foot}{fg=black, bg=white}
%\vskip1em \tiny \insertsectionnavigationhorizontal{1\paperwidth}{\hspace{0.50\paperwidth}}{}
%}
%%------------------------------------------
%% \begin{frame}\frametitle{}
%% \begin{columns}
%% \label{Appendix}
%% \column{1\linewidth}
%% \centering
%% {\Large \alert{Appendix}}
%% \end{columns}
%% \end{frame}
%%------------------------------------------
%\begin{frame}[allowframebreaks]{References}
%\footnotesize
%\bibliographystyle{$BIB_STYLE}
%\bibliography{$BIBFILE}
%\end{frame}

\end{document}
