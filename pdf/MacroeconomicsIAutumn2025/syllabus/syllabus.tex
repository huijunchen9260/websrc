\documentclass[12pt]{article}

% \usepackage[style=authoryear,maxbibnames=9,maxcitenames=2,uniquelist=false,backend=biber,doi=false,url=false]{biblatex}
\usepackage{natbib}
% \addbibresource{$BIB} % bibtex location
% \renewcommand*{\nameyeardelim}{\addcomma\space} % have comma in parencite

\usepackage{xcolor}
 \usepackage{amsmath}
\newcommand{\tuple}[1]{ \langle #1 \rangle }
%\usepackage{automata}
\usepackage{times}
\usepackage{ltablex}
\usepackage{natbib}
\usepackage{fontspec}
\usepackage{luatexja}
\usepackage[mathscr]{euscript}

%%%%%% Template
\usepackage{hyperref}
\hypersetup{colorlinks=true,allcolors=blue}

\usepackage{vmargin}
\setpapersize{USletter}
\setmarginsrb{1.0in}{1.0in}{1.0in}{0.6in}{0pt}{0pt}{0pt}{0.4in}

% HOW TO USE THE ABOVE:
%\setmarginsrb{leftmargin}{topmargin}{rightmargin}{bottommargin}{headheight}{headsep}{footheight}{footskip}
%\raggedbottom
% paragraphs indent & skip:
\parindent  0.3cm
\parskip    -0.01cm

\usepackage{tikz}
\usetikzlibrary{backgrounds}

% hyphenation:
\sloppy

% notes-style paragraph spacing and indentation:
\usepackage{parskip}
\setlength{\parindent}{0cm}

% let derivations break across pages
\allowdisplaybreaks

\def\blue{\color{blue}}
\def\orange{\color{orange}}

\def\qqquad{\quad\qquad}
\def\qqqquad{\qquad\qquad}

\setmainfont{Crimson Pro Light}[
  ItalicFont={* Italic},
  BoldFont={Crimson Pro Medium},
  BoldItalicFont={Crimson Pro Medium Italic}]
\setsansfont{Crimson Pro Light}[
  ItalicFont={* Italic},
  BoldFont={Crimson Pro Medium},
  BoldItalicFont={Crimson Pro Medium Italic}]

%%%%%%%%%%%%%%%%%%%%%%%%%%%%%%%%%%%%%%%%%%%%%%%%%%%%%%%%%%%%%%%%%%%%%%%%%%%%%%%%
%%%%%%%%%%%%%%%%%%%%%%%%%%%%%%%%%%%%%%%%%%%%%%%%%%%%%%%%%%%%%%%%%%%%%%%%%%%%%%%%
\begin{document}

\centerline{\huge\bf Syllabus: Macroeconomics I}
\medskip
\centerline{\LARGE \bf Autumn 2025}
\medskip
\centerline{\Large Instructor: Hui-Jun Chen}
\centerline{Last Update: \today}
\centerline{Lastest Version: \href{https://huijunchen9260.github.io/pdf/MacroeconomicsIAutumn2025/syllabus/syllabus.pdf}{Click Here}}

\medskip

% \tableofcontents

% \newpage

\section*{Course Overview}
\addcontentsline{toc}{section}{Course Overview}
\begin{itemize}

    \item Course website:
    \begin{itemize}
        \item Materials: \href{https://huijunchen9260.github.io/MacroeconomicsIAutumn2025.html}{Webpage}
    \end{itemize}
    \item Meeting Time: Thursday 10:10 - 13:00 (R34n)
    \item Location: TSMC Building 224
    \item Office: TSMC Building 729-B
    \item Email address: \href{huijunchen@mx.nthu.edu.tw}{huijunchen@mx.nthu.edu.tw}
    \item Please do not hesitate to email me and set an appointment outside of regular office hour. To get quicker email reply, I would prefer you to:
    \begin{enumerate}
        \item Use \texttt{[Macro]} at the beginning of your subject title.
        \begin{itemize}
            \item example title: \texttt{[Macro] Question regarding Extra credit}
        \end{itemize}
    \end{enumerate}
    \item I will reply your email within \textit{2 business day}.
    \item Office hour: By appointment
    \item Teaching Assistant:
    \begin{enumerate}
        \item 陳建廷(email: \href{ericchen904230331@gmail.com}{ericchen904230331@gmail.com})
        \item 曾子齊(email: \href{freeboard2001@gmail.com}{freeboard2001@gmail.com})
    \end{enumerate}
    \item TA Office hour: 15:20 - 16:20, Tuesday at Room 515, TSMC Building
\end{itemize}

\newpage

\section*{Grades}
\addcontentsline{toc}{section}{Grades}

\newlength\q
\setlength\q{\dimexpr .5\textwidth -2\tabcolsep}
\begin{tabular}{|p{\q}|p{\q}|}
    \hline
    Categories  & Points \\
    \hline
    \hline
    Problem sets on course material   & 20 points \\
    \hline
    Midterm Exam I & 20 points \\
    \hline
    Midterm Exam II & 20 points \\
    \hline
    Final Exam & 30 points \\
    \hline
    Attendance & 10 points \\
    \hline
    Total & 100 points \\
    \hline
\end{tabular}
\textit{See course schedule, below, for due dates} \\


\section*{Grading Policy}
\addcontentsline{toc}{section}{Grading Policy}

% \subsection*{Quizzes / Exams}
% \addcontentsline{toc}{subsection}{Quizzes / Exams}

% Weekly quizzes in this class: \underline{Calculus materials}.
% You will have \textbf{unlimited} attempts and \textbf{unlimited time} per attempt for quizzes of Calculus materials.
% When calculating the final grade, I will \textbf{drop two quiz with the lowest grade} in each category (except Quiz on Calculus, Ch. 10 \& 11).

% The exact due date and time for quizzes and exams should refer to the schedule below and the setting on Carmen.
% In principle, all quizzes are due on \textbf{Sunday 11:59pm}, and the answer is available on \textbf{next Monday}.

% Late quizzes are not accepted, unless you have formal excuse (require formal documentation, and the \textbf{instrutor still has rights to decide whether to extend the quiz / exams for this excuse}).

% Final exam are cumulative, so the content from the midterm is also included in final exam.

% \section*{Quizzes and Examinations Integrity Policies}
% \addcontentsline{toc}{section}{Quizzes and Examinations Integrity Policies}
\section*{Examinations Integrity Policies}
\addcontentsline{toc}{section}{Examinations Integrity Policies}

\textbf{Problem Sets}: Discussions are \textbf{encouraged}, but each person must hand in their own answer sheets

\textbf{Examinations}: Discussions are \textbf{forbidden}, either face to face or via online discussion board / Social media.

% \subsection*{Extra credits}

% Extra credits: At the end of the semester, I will have you do the Student Evaluation of Instruction (SEI). If I get $80\%$ response rate on SEI by the end of the semester, everyone will get 2 points of extra credits.

\subsection*{Curving}
\addcontentsline{toc}{subsection}{Curving}

If less than $40\%$ of the students get A- or above, I will add some points to everybody until $40\%$ of the students get A- or above, but I don't expect this to occur.

\subsection*{Problem Sets}
\label{sub:Problem_Sets}
\addcontentsline{toc}{subsection}{Problem Sets}

There are four problem sets and each of them worth 5 points.
The way to calculate semester grade on problem set is by $ \frac{ \texttt{correct answers} }{ \texttt{total number of questions} } \times 5 $.
For example, if there are 36 questions and you answer 30 correctly, then you will get $ \frac{30}{36} \times 5 \approx 4.16$ points in semester grade.


Problem sets will be answered on paper, and will be submitted by the class time to TA.

Note that overdue assignment will \textbf{NOT} be accepted.

\subsection*{Exam}
\label{sub:Exam}
\addcontentsline{toc}{subsection}{Exam}

There are two midterm exams and each of them worth 20 points.
There is one final exam and it worths 30 points.
The way to calculate semester grade on exams is the similar to the Problem Sets if not otherwise noted.

\subsection*{Attendance}
\label{sub:Attendance}
\addcontentsline{toc}{subsection}{Attendance}

If there are more than $ 50\% $ of students attend the class, then I will not take attendance.
If I have not taken attendance until the end of semester, then every students will be granted the entirety of the attendance points.
If I start to take attendance, then I will spread $ 5 $ to $ 10 $ attendance check across the rest of the semester, and the attendance points will be recorded accordingly.


% \subsection*{Midterm \& Final weight change}
% \label{sub:Midterm____Final_weight_change}
% \addcontentsline{toc}{subsection}{Midterm \& Final weight change}

% After the midterm exam, there will be a survey on Carmen asking whether you want to change the weight between midterm and final from $ 35 $ points each to $ 10 $ points for midterm and $ 60 $ points for final.
% This survey will due \textbf{before} the final exam date, and you cannot change the weight between midterm and final after you took the final exam.


% \section*{Course Attendance Policy}

% According to the new ASC policy, \textbf{strictly masking} is required, which means you should wear your mask to cover your mouth and nose.
% Also, the students \textbf{are not allowed to eat or drink} in class.

% The definition of in-person class also redefined by new ASC policy.
% Up to \textbf{24\%} of the class meetings can be online.
% If I cannot teach the course in person due to personal issue, I will make an announcement on \href{https://osu.instructure.com/courses/114824}{Carmen}, and the class will be held online in the \href{https://osu.zoom.us/j/2532324996?pwd=c2cweEphWFMvTVZreHJ0MHNRNUdodz09}{zoom link}.

% The instructor is also required to make reasonable accommodations for students who cannot attend the classes for a period of time, either because of sickness or quarantine.
% My arrangement is to open all my class recordings back in the summer in the \href{https://huijunchen9260.github.io/PrincipleMacroSpring2022.html}{Webpage} so that students who cannot come to class can also reach out the course materials by watching the recorded videos.
% Also, since the office hour is online, those students who cannot attend the in-person meeting can also ask me question on the \href{https://osu.zoom.us/j/2532324996?pwd=c2cweEphWFMvTVZreHJ0MHNRNUdodz09}{zoom link}.

\section*{Tentative Course Schedule}
\addcontentsline{toc}{section}{Tentative Course Schedule}

\newlength\bb
\setlength\bb{\dimexpr .08\textwidth -2\tabcolsep}
\newlength\qq
\setlength\qq{\dimexpr .14\textwidth -2\tabcolsep}
\newlength\rr
\setlength\rr{\dimexpr .3\textwidth -2\tabcolsep}
\newlength\pp
\setlength\pp{\dimexpr .5\textwidth -2\tabcolsep}
\begin{tabular}{|p{\bb}|p{\bb}|p{\pp}|p{\rr}|}
    \hline
        Week & Day & Topics and Readings & Deadlines \\
    \hline
    \hline
        1
        &
        9/4
        &
        Topic: Introduction
        \newline
        Topic: Measurement I
        &
    \\
    \hline
        2
        &
        9/11
        &
        Topic: Measurement II
        \newline
        Topic: Consumer Preference I
        &
    \\
    \hline
        3
        &
        9/18
        &
        Topic: Consumer Preference II
        \newline
        Topic: Examples
        &
        Problem Set 1
    \\
    \hline
        4
        &
        9/25
        &
        Topic: Firms
        \newline
        Topic: Competitive Equilibrium
        &
    \\
    \hline
        5
        &
        10/2
        &
        Topic: Social Planer's Problem
        \newline
        Topic: Examples
        &
        Problem Set 2
    \\
    \hline
        6
        &
        10/9
        &
        Midterm I
        &
    \\
    \hline
        7
        &
        10/16
        &
        Midterm Recap
        \newline
        Topic: Distorting Taxes
        &
    \\
    \hline
        8
        &
        10/23
        &
        Topic: Two Period Consumer Problem
        \newline
        Topic: Two Period Equilibrium
        &
    \\
    \hline
        9
        &
        10/30
        &
        Topic: RBC Model Part 1: Consumer
        \newline
        Topic: RBC Model Part 2: Firm
        &
    \\
    \hline
        10
        &
        11/06
        &
        Topic: RBC Model Part 3: Competitive Equilibrium
        \newline
        Topic: RBC Model Part 4: Examples
        &
    \\
    \hline
        11
        &
        11/13
        &
        Topic: RBC Model Part 5: Applications
        \newline
        Topic: Solow Model and Dynamic Programming
        \newline
        \textbf{Online class: Going to the US for conference}
        &
        Problem Set 3
    \\
    \hline
        12
        &
        11/20
        &
        Midterm II
        &
    \\
    \hline
        13
        &
        11/27
        &
        Topic: Asset Pricing: Endowment Economy
        \newline
        Topic: Asset Pricing: Production Economy
        &
    \\
    \hline
        14
        &
        12/04
        &
        Customized Class
        &
        Problem Set 4
    \\
    \hline
        15
        &
        12/11
        &
        Customized Class
        \newline
        Final Review
        &
    \\
    \hline
        16
        &
        12/18
        &
        Final Exam
        &
    \\
    \hline
\end{tabular}

\section*{Grading scale}

See \url{https://registra.site.nthu.edu.tw/var/file/211/1211/img/609/grade-plan.pdf} for NTHU definition.

\end{document}

