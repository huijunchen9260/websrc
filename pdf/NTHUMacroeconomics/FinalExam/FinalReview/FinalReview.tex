\documentclass[11pt,aspectratio=169,usenames,dvipsnames]{beamer}
\usetheme[nomail, amurmapleblack]{Amurmaple}

\usepackage{threeparttable}
\usepackage{booktabs}
\usepackage{xcolor} % For custom colors
\usepackage{tikz} % For styling enumerate numbers
\usepackage{tcolorbox} % For colored box styling
\usepackage{amsmath, amsfonts, amssymb, amsthm} % Math related
\usepackage{natbib}
\usepackage{fontspec}
\usepackage{luatexja}
\usepackage[mathscr]{euscript}

% ---------------- %
% color definition %
% ---------------- %
\definecolor{main}{HTML}{23373B}
\definecolor{pink}{RGB}{180, 50, 110}
\definecolor{orange}{HTML}{FF8000}
\definecolor{red}{HTML}{990000}
\definecolor{blue}{HTML}{004C99}
\definecolor{lightgray}{HTML}{E7E7E7}
\definecolor{gray}{RGB}{90, 90, 90}

\newcommand{\pink}[1]{\textcolor{pink}{#1}}
\newcommand{\orange}[1]{\textcolor{orange}{#1}}
\newcommand{\red}[1]{\textcolor{red}{#1}}
\newcommand{\blue}[1]{\textcolor{blue}{#1}}
\newcommand{\green}[1]{\textcolor{OliveGreen}{#1}}
\newcommand{\magenta}[1]{\textcolor{magenta}{#1}}
\newcommand{\gray}[1]{\textcolor{gray}{#1}}
\newcommand{\purple}[1]{\textcolor{purple}{#1}}
\definecolor{yellow}{HTML}{EDB120}

% \setbeamercolor{alerted text}{fg=blue}

%%% automatically add spaces into enumerate and itemize environment
\let\tempone\itemize
\let\temptwo\enditemize
\renewenvironment{itemize}{\tempone\addtolength{\itemsep}{\fill}}{\temptwo}
\let\tempa\enumerate
\let\tempb\endenumerate
\renewenvironment{enumerate}{\tempa\addtolength{\itemsep}{\fill}}{\tempb}

\setsansfont{Alegreya Sans Light}[
  ItalicFont={* Italic},
  BoldFont={Alegreya Sans Medium},
  BoldItalicFont={Alegreya Sans Medium Italic}]

\usepackage[mode=tex]{standalone}
\usepackage{tikz}
\usetikzlibrary{decorations}
\usetikzlibrary{decorations.pathreplacing, intersections}
\usepackage{pgfplots}
\usetikzlibrary{calc,positioning}
\usepgfplotslibrary{fillbetween}
\pgfplotsset{compat=newest, scale only axis, width = 10cm}

% --------------------------- %
% Section title page with toc %
% --------------------------- %
\setbeamertemplate{subsection page}{%
    \usebeamertemplate*{section page}
}
\setbeamertemplate{section in toc}[square]
\setbeamertemplate{subsection in toc}[square]
\AtBeginSection[]{
% \sepframe
\begin{frame}[noframenumbering]{Outline}
    % \tableofcontents[currentsection]
    \tableofcontents[currentsection, currentsubsection]
\end{frame}
}
\AtBeginSubsection[]{
  \begin{frame}[noframenumbering]{Outline}
    \tableofcontents[currentsection, currentsubsection]
  \end{frame}
}

% ------------ %
% beamerbutton %
% ------------ %
\newcommand{\goto}[2]{\hyperlink{#2}{\beamergotobutton{#1}}}
\newcommand{\return}[2]{\hyperlink{#2}{\beamerreturnbutton{#1}}}
\newcommand{\extgoto}[2]{\href{#2}{\beamergotobutton{#1}}}

\title[Final Review]{Intermediate Macro Theory}
\subtitle{FInal exam review}
\author{Hui-Jun Chen}
\institute{The Ohio State University}
\webpage{https://huijunchen9260.github.io/}
\date{\today}

\begin{document}

% Title Page
\begin{frame}[noframenumbering]
    \titlepage
\end{frame}

\section{Human Capital Accumulation: (Lucas, 1988)}
\label{sec:Human_Capital_Accumulation___Lucas__1988_}

\begin{frame}{Environment}
\label{slide:Environment}
\begin{itemize}
    \item Spend \alert{time} in education to accumulate human capital
    \item Utility: $ \displaystyle U(C, C') = u(C) + u(C') $
    \item $ \phi $ fraction of one unit of time endowment goes to \alert{work}
    \item $ 1 - \phi $ fraction of one unit of time endowment goes to \alert{education}
    \item Human capital law of motion: $ \displaystyle H' = H + (1 - \varphi) H = (2 - \varphi)H$
    \item Physical capital law of motion: $ \displaystyle K' = (1 - \delta)K + I $
    \item Production: $ \displaystyle Y = K^{\alpha} (\varphi H)^{1 - \alpha} $; $ \displaystyle Y' = K'^{\alpha} (\varphi' H')^{1 - \alpha} $
\end{itemize}

\end{frame}

\begin{frame}{Constructing Optimization Problem}
\label{slide:Constructing_Optimization_Problem}
    \begin{itemize}
        \item Labor income: $ w \varphi H $
        \item Capital income: $ \displaystyle r K $
        \item Budget constraints: $ \displaystyle C + I \le w \varphi H + r K + \pi $
        \item Profit: $ \displaystyle \pi = Y - w \varphi H - r K $
        \item Plug profit into budget constraints to get aggregate resource constraints (Income-Expenditure Identity)
    \end{itemize}
\end{frame}

\begin{frame}{Social Planner's Problem}
\label{slide:Social_Planner_s_Problem}
    \[
        \begin{split}
            \max_{\varphi, K'}
                & \qquad u(K^{\alpha} (\varphi H)^{1 - \alpha} + (1 - \delta)K - K') + u((K')^{\alpha}(\varphi' (2 - \varphi) H)^{1-\alpha})
            \\
        \end{split}
    \]
    \begin{itemize}
        \item $ \varphi' = 1$ as there's no third period
        \item First order conditions yield
        \[
            \begin{split}
                [\varphi]:
                    & \qquad u'(C) (1 - \alpha) K^{\alpha} \varphi^{1-\alpha} H^{-\alpha} = u'(C') (1 - \alpha) (K')^{\alpha} ((2 - \varphi))^{-\alpha} \varphi H^{-\alpha}
                \\
                    & \qquad MRS_{C, C'} = \frac{u'(C)}{u'(C')} = \left(
                        \frac{K'}{K}
                    \right)^{\alpha} \left(
                        \frac{2 - \varphi}{\varphi}
                    \right)^{-\alpha}
                \\
                [K']:
                    & \qquad u'(C) = u'(C') \alpha (K')^{\alpha-1} ((2 - \varphi)H)^{1-\alpha}
                \\
                    & \qquad MRS_{C, C'} = \frac{u'(C)}{u'(C')} = \alpha (K')^{\alpha-1} ((2 - \varphi)H)^{1-\alpha}
                \\
            \end{split}
        \]
    \end{itemize}
\end{frame}

\begin{frame}{Relationship between two intertemporal assets}
\label{slide:Relationship_between_two_intertemporal_assets}
\begin{itemize}
    \item Two equation equates,
    \[
        \left(
                        \frac{K'}{K}
                    \right)^{\alpha} \left(
                        \frac{2 - \varphi}{\varphi}
                    \right)^{-\alpha}
                    = \alpha (K')^{\alpha-1} ((2 - \varphi)H)^{1-\alpha}
    \]
    \item Simplify,
    \[
        \frac{\varphi^{\alpha}}{2 - \varphi} K' = \alpha K^{\alpha} H^{1-\alpha}
    \]
    Notice the RHS is constant.
    \item This is another expressions for optimal investment schedule. The return on human capital, which denotes by decrease in $ \psi $, and the return on physical capital, $ K' $ should be equally favarable in equilibrium.
\end{itemize}
\end{frame}


\section{Solow Model with Labor Growth}
\label{sec:Solow_Model_with_Labor_Growth}

\begin{frame}{Overview}
\label{slide:Overview}
\begin{itemize}
    \item Consider a Solow model with economic growth
    \item Labor productivity grows at rate $ \gamma $: $ \displaystyle X_{t+1} = (1 + \gamma) X_{t} $
    \item Population grows at rate $ n $: $ \displaystyle L_{t+1} = (1 + n) n_{t} $
    \item Effective labor force: $ \displaystyle N_{t} = X_{t} L_{t} $
    \item Production: $ \displaystyle Y_{t} = A K_{t}^{\alpha} N_{t}^{1-\alpha} $
    \item Consumption is residual of saving, $ \displaystyle C_{t} = (1 - s) Y_{t} $
    \item Full depreciation on capital, $ K_{t+1} = I_{t} $
    \item Aggregate resource constraint, $ \displaystyle C_{t} + I_{t} = Y_{t} $
\end{itemize}
\end{frame}


\begin{frame}{Constructing efficiency unit of capital}
\label{slide:Constructing_capital_per_effective_labor}
    \begin{itemize}
        \item Investment is given by $ \displaystyle K_{t+1} = I_{t} = s Y_{t} $
        \item Growth of effective labor, $ \displaystyle \frac{N_{t+1}}{N_{t}} = \frac{X_{t+1} L_{t+1}}{X_{t}L_{t}} = (1 + \gamma)(1 + n) $
        \item Future capital to current effective labor ratio, $ \displaystyle \frac{K_{t+1}}{N_{t}} = \frac{s Y_{t}}{N_{t}} $
        \item Let $ k_{t} = \frac{K_{t+1}}{N_{t+1}} $ denotes the efficiency unit of capital, or capital-labor ratio. The law of motion of $ k_{t} $ is
        \[
        \begin{split}
            \frac{s Y_{t}}{N_{t}}
                & = \frac{K_{t+1}}{N_{t}} = \frac{K_{t+1}}{N_{t+1}} \frac{N_{t+1}}{N_{t}}
            \\
            \frac{s A K_{t}^{\alpha} N_{t}^{1-\alpha}}{N_{t}}
                & = k_{t+1} (1+\gamma)(1+n)
            \\
            k_{t+1}
                & = \frac{s A}{(1+\gamma)(1+n)} k_{t}^{\alpha}
            \\
        \end{split}
        \]
    \end{itemize}
\end{frame}

\begin{frame}{Steady state}
\label{slide:Steady_state}
    \begin{itemize}
        \item In steady state, $ k_{t+1} = k_{t} = k^{*} $,
        \[
            \begin{split}
                k^{*}
                    & = \frac{s A}{(1+\gamma)(1+n)} k^{*\alpha}
                \\
                k^{*}
                    & = \left(
                        \frac{s A}{(1+\gamma)(1+n)}
                    \right)^{1 - \alpha}
                \\
            \end{split}
        \]
        \item Comparing two economy one with large population growth (larger $ n $) and larger saving rate ( larger $ s $) than the other, but have constant ratio between the two will exhibit the same efficiency in production, as their efficiency unit of capital should be the same. Just one is larger in scale/size.
    \end{itemize}
\end{frame}

\end{document}
