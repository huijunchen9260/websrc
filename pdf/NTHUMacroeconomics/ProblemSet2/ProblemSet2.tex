\documentclass[14pt]{extarticle}

% \usepackage[style=authoryear,maxbibnames=9,maxcitenames=2,uniquelist=false,backend=biber,doi=false,url=false]{biblatex}
% \addbibresource{$BIB} % bibtex location
% \renewcommand*{\nameyeardelim}{\addcomma\space} % have comma in parencite
\usepackage{natbib}

\usepackage{xcolor}
\usepackage{amsmath}
\newcommand{\tuple}[1]{ \langle #1 \rangle }
%\usepackage{automata}
\usepackage{times}
\usepackage{ltablex}
\usepackage{tasks}

%%%%%% Template
\usepackage{hyperref}
\hypersetup{colorlinks=true,allcolors=blue}

\usepackage{vmargin}
\setpapersize{USletter}
\setmarginsrb{1.0in}{1.0in}{1.0in}{0.6in}{0pt}{0pt}{0pt}{0.4in}

% HOW TO USE THE ABOVE:
%\setmarginsrb{leftmargin}{topmargin}{rightmargin}{bottommargin}{headheight}{headsep}{footheight}{footskip}
%\raggedbottom
% paragraphs indent & skip:
\parindent  0.3cm
\parskip    -0.01cm

\usepackage{tikz}
\usetikzlibrary{backgrounds}

% hyphenation:
% \hyphenpenalty=10000 % no hyphen
% \exhyphenpenalty=10000 % no hyphen
\sloppy

% notes-style paragraph spacing and indentation:
\usepackage{parskip}
\setlength{\parindent}{0cm}

% let derivations break across pages
\allowdisplaybreaks

\newcommand{\orange}[1]{\textcolor{orange}{#1}}
\newcommand{\blue}[1]{\textcolor{blue}{#1}}
\newcommand{\red}[1]{\textcolor{red}{#1}}
\newcommand{\freq}[1]{{\bf \sf F}(#1)}
\newcommand{\datafreq}[2]{{{\bf \sf F}_{#1}(#2)}}

\def\qqquad{\quad\qquad}
\def\qqqquad{\qquad\qquad}

%%%%%%%%%%%%%%%%%%%%%%%%%%%%%%%%%%%%%%%%%%%%%%%%%%%%%%%%%%%%%%%%%%%%%%%%%%%%%%%%
%%%%%%%%%%%%%%%%%%%%%%%%%%%%%%%%%%%%%%%%%%%%%%%%%%%%%%%%%%%%%%%%%%%%%%%%%%%%%%%%

% fill-in-blank question style, found in https://tex.stackexchange.com/a/505089

\usepackage{ifthen}
\usepackage{tocloft}
\usepackage{exercise}
% \usepackage{xcolor}

% Set the Show Answers Boolean
\newboolean{showAns}
\setboolean{showAns}{false}
\newcommand{\showAns}{\setboolean{showAns}{true}}

% The length of the Answer line
\newlength{\answerlength}
\newcommand{\anslen}[1]{\settowidth{\answerlength}{#1}}

% ans command that indicates space for an answer or shows the answer in red
\newcommand{\ans}[1]{\settowidth{\answerlength}{\hspace{2ex}#1\hspace{2ex}}%
    \ifthenelse{\boolean{showAns}}%
        {\textcolor{red}{\underline{\hspace{2ex}#1\hspace{2ex}}}}%
        {\underline{\hspace{\answerlength}}}}%

\newcommand{\details}[1]{\settowidth{\answerlength}{#1}%
    \ifthenelse{\boolean{showAns}}%
        {\\ \textcolor{blue}{#1}}%
        {}}%

% Formatting how multiple choices Questions are formated.
\settasks{label=(\Alph*), label-width=30pt}


% Some commands for the Exercise Question package
\renewcommand{\QuestionNB}{\Large\protect\textcircled{\small\bfseries\arabic{Question}}\ }
\renewcommand{\ExerciseHeader}{} %no header
\renewcommand{\QuestionBefore}{3ex} %Space above each Q
\setlength{\QuestionIndent}{8pt} % Indent after Q number


% To create the list of answers with tocloft...
\newcommand{\listanswername}{Answers}
\newlistof[Question]{answer}{Answers}{\listanswername}

% Creates a TOC for Answers
\newcounter{prevQ}
\newcommand{\answer}[1]{\refstepcounter{answer}%
\ans{#1}%
\ifnum\theQuestion=\theprevQ%
        \addcontentsline{Answers}{answer}{\protect\numberline{}#1}% don't include the Q number
        \else%
        \addcontentsline{Answers}{answer}{\protect\numberline{\theQuestion}#1}%
        \setcounter{prevQ}{\value{Question}}%
        \fi%
        }%

% \hyphenpenalty=10000 % no hyphen
% \exhyphenpenalty=10000 % no hyphen
\sloppy              % hyphen

\newcommand{\HRule}{\rule{\linewidth}{0.5mm}}
\newcommand{\Hrule}{\rule{\linewidth}{0.3mm}}

%tocloft formatting listofanswers
\renewcommand{\cftAnswerstitlefont}{\bfseries\large}
\renewcommand{\cftanswerdotsep}{\cftnodots}
\cftpagenumbersoff{answer}
\addtolength{\cftanswernumwidth}{10pt}

\makeatletter% since there's an at-sign (@) in the command name
\renewcommand{\@maketitle}{%
  \parindent=0pt% don't indent paragraphs in the title block
  \centering
  {\Large \bfseries\textsc{\@title}} \\
  \vspace{5pt}
  {\large \textit{\@author}} \\
  \HRule \\
  \vspace{1em}
}
\makeatother% resets the meaning of the at-sign (@)

\title{Problem Set 2}
\author{Hui-Jun Chen}

\begin{document}

\maketitle

% \showAns
% \listofanswer

% \section*{Instruction}

% % Due at 11:59 PM (Eastern Time) on Sunday, June 14, 2022.

% Please answer this problem set on Carmen quizzes ``Problem Set 2''. In the following problems, the part that is in \textbf{\red{red and bold}} are the order of questions that should be answered on Carmen quizzes.

\begin{Exercise}

\section*{Problem 1}

Remember the Example in Lecture 8.

    Consumer: $ \max_{C, l} \ln C + \ln l \quad \text{subject to} \quad C \le w( 1-l ) + \pi $
    %
    \begin{align}
        \text{FOC} \quad
            & \frac{C}{l} = w
            \label{eq:consumerFOC}
        \\
        \text{Binding budget constraint} \quad
            & C = w ( 1-l ) + \pi
            \label{eq:binding_budget}
        \\
        \text{Time constraint} \quad
            & N^{s} = 1 - l
            \label{eq:time_budget}
    \end{align}
    %

    Firm: $ \max_{N^{d}} ( N^{d} )^{\frac{1}{2}} - w N^{d} $
    %
    \begin{align}
        \text{FOC} \quad
            & \frac{1}{2} ( N^{d} )^{- \frac{1}{2}} = w
            \label{eq:firmFOC}
        \\
        \text{Output definition} \quad
            & Y = ( N^{d} )^{\frac{1}{2}}
            \label{eq:outputDef}
        \\
        \text{Profit definition} \quad
            & \pi = Y - w N^{d}
            \label{eq:profitDef}
    \end{align}
    %
    Market clear:
    %
    \begin{align}
        N^{s} & = N^{d}
        \label{eq:laborClear}
    \end{align}
    %

Fill the following blanks for the step-by-step guide for algebraic calculation:

    Step 1: Impose Market clear condition, so shrink all $ 7 $ equations to \textbf{\red{\underline{\quad Q1 \quad}}} equations

    \Question
    \answer{C}
    \begin{tasks}(4)
        \task 4
        \task 5
        \task 6
        \task 7
    \end{tasks}

    Consumer: $ \max_{C, l} \ln C + \ln l \quad \text{subject to} \quad C \le w( 1-l ) + \pi $
    %
    \begin{align}
        \text{FOC} \quad
            & \frac{C}{l} = w
            \label{eq:consumerFOC}
        \\
        \text{Binding budget constraint} \quad
            & C = w ( 1-l ) + \pi
            \label{eq:binding_budget}
        \\
        \text{Time constraint} \quad
            & N = 1 - l
            \label{eq:time_budget}
    \end{align}
    %

    Firm: $ \max_{N} ( N )^{\frac{1}{2}} - w N $
    %
    \begin{align}
        \text{FOC} \quad
            & \frac{1}{2} ( N )^{- \frac{1}{2}} = w
            \label{eq:firmFOC}
        \\
        \text{Output definition} \quad
            & Y = ( N )^{\frac{1}{2}}
            \label{eq:outputDef}
        \\
        \text{Profit definition} \quad
            & \pi = Y - w N
            \label{eq:profitDef}
    \end{align}
    %
    Step 2: replace $ l $ in terms of $ N $ using $ l = 1-N $

    Consumer: $ \max_{C, l} \ln C + \ln l \quad \text{subject to} \quad C \le w( 1-l ) + \pi $
    %
    \begin{align}
        \text{FOC} \quad
            & \frac{C}{(\textbf{\red{\underline{\quad Q2 \quad}}})} = w
            \label{eq:consumerFOC}
        \\
        \text{Binding budget constraint} \quad
            & C = w (\textbf{\red{\underline{\quad Q3 \quad}}}) + \pi
            \label{eq:binding_budget}
    \end{align}
    %

    \Question
    \answer{A}
    \begin{tasks}(4)
        \task $1 - N$
        \task $ 2  - N $
        \task $ N - 1 $
        \task $ 2 - N $
    \end{tasks}

    \Question
    \answer{D}
    \begin{tasks}(4)
        \task $ N - 3 $
        \task $ N - 2 $
        \task $ N - 1 $
        \task $ N $
    \end{tasks}

    Firm: $ \max_{N} ( N )^{\frac{1}{2}} - w N $
    %
    \begin{align}
        \text{FOC} \quad
            & \frac{1}{2} ( N )^{- \frac{1}{2}} = w
            \label{eq:firmFOC}
        \\
        \text{Output definition} \quad
            & Y = ( N )^{\frac{1}{2}}
            \label{eq:outputDef}
        \\
        \text{Profit definition} \quad
            & \pi = Y - w N
            \label{eq:profitDef}
    \end{align}
    %
    Step 3: replace $ \pi $ and $ Y $ as $ N $

    Consumer: $ \max_{C, l} \ln C + \ln l \quad \text{subject to} \quad C \le w( 1-l ) + \pi $
    %
    \begin{align}
        \text{FOC} \quad
            & \frac{C}{(\textbf{\red{\underline{\quad Q2 \quad}}})} = w
            \label{eq:consumerFOC}
        \\
        \text{Binding budget constraint} \quad
            & C = w (\textbf{\red{\underline{\quad Q3 \quad}}}) + \pi
            \label{eq:binding_budget}
    \end{align}
    %

    Firm: $ \max_{N} ( N )^{\frac{1}{2}} - w N $
    %
    \begin{align}
        \text{FOC} \quad
            & \frac{1}{2} ( N )^{- \frac{1}{2}} = w
            \label{eq:firmFOC}
        \\
        \text{Profit definition} \quad
            & \pi = (\textbf{\red{\underline{\quad Q4 \quad}}}) - w N
            \label{eq:profitDef}
    \end{align}
    %

    \Question
    \answer{C}
    \begin{tasks}(4)
        \task $ N $
        \task $ 2 N $
        \task $ N^{\frac{1}{2}} $
        \task $ N^{\frac{1}{4}} $
    \end{tasks}

    Step 4: Substitute $ \pi( N ) $ into Binding budget constraint and get
    %
    \begin{equation}
    \label{eq:C_as_function_of_N}
        C = (\textbf{\red{\underline{\quad Q5 \quad}}})
    \end{equation}
    %

    \Question
    \answer{B}
    \begin{tasks}(4)
        \task $ N$
        \task $ N^{\frac{1}{2}} $
        \task $ N^{\frac{1}{4}} $
        \task $ N^{\frac{1}{8}} $
    \end{tasks}

    Step 5: With consumer's FOC and firm's FOC both equate to $ w $, we can get another expression of $ C $:
    %
    \begin{equation}
    \label{eq:C_as_function_of_N_ver_2}
        C = (\textbf{\red{\underline{\quad Q2 \quad}}}) \times  (\textbf{\red{\underline{\quad Q6 \quad}}})
    \end{equation}
    %

    \Question
    \answer{A}
    \begin{tasks}(4)
        \task $ \frac{1}{2} N^{- \frac{1}{2}} $
        \task $ \frac{1}{2} N $
        \task $ \frac{1}{4} N^{- \frac{1}{2}} $
        \task $ \frac{1}{2} N $
    \end{tasks}

    Step 6: Let \eqref{eq:C_as_function_of_N} equate \eqref{eq:C_as_function_of_N_ver_2} and we get $ N $ as
    %
    \begin{equation}
    \label{eq:Nvalue}
        N = (\textbf{\red{\underline{\quad Q7 \quad}}})
    \end{equation}
    %

    \Question
    \answer{C}
    \begin{tasks}(4)
        \task $ 1 $
        \task $ \frac{1}{2} $
        \task $ \frac{1}{3} $
        \task $ \frac{1}{4} $
    \end{tasks}

    Step 7: Trace back to all unknowns given the value of $ N $, we get
    %
    \begin{align}
        C
            & = (\textbf{\red{\underline{\quad Q8 \quad}}})
        \\
        l
            & = (\textbf{\red{\underline{\quad Q9 \quad}}})
        \\
        Y
            & = (\textbf{\red{\underline{\quad Q10 \quad}}})
        \\
        \pi
            & = (\textbf{\red{\underline{\quad Q11 \quad}}})
        \\
        w
            & = (\textbf{\red{\underline{\quad Q12 \quad}}})
    \end{align}
    %

    \Question
    \answer{A}
    \begin{tasks}(4)
        \task $ \sqrt{\frac{1}{3}} $
        \task $ \sqrt{\frac{1}{4}} $
        \task $ \sqrt{\frac{1}{5}} $
        \task $ \sqrt{\frac{1}{6}} $
    \end{tasks}

    \Question
    \answer{D}
    \begin{tasks}(4)
        \task $ \frac{3}{4} $
        \task $ \frac{1}{3} $
        \task $ \frac{1}{4} $
        \task $ \frac{2}{3} $
    \end{tasks}

    \Question
    \answer{A}
    \begin{tasks}(4)
        \task $ \sqrt{\frac{1}{3}} $
        \task $ \sqrt{\frac{1}{4}} $
        \task $ \sqrt{\frac{1}{5}} $
        \task $ \sqrt{\frac{1}{6}} $
    \end{tasks}

    \Question
    \answer{B}
    \begin{tasks}(4)
        \task $ \sqrt{\frac{1}{3}} - \frac{1}{3} \sqrt{3}$
        \task $ \sqrt{\frac{1}{3}} - \frac{1}{6} \sqrt{3}$
        \task $ \sqrt{\frac{1}{6}} - \frac{1}{3} \sqrt{3}$
        \task $ \sqrt{\frac{1}{6}} - \frac{1}{6} \sqrt{3}$
    \end{tasks}

    \Question
    \answer{C}
    \begin{tasks}(4)
        \task $ \frac{1}{3} \sqrt{3} $
        \task $ \frac{1}{2} \sqrt{2} $
        \task $ \frac{1}{2} \sqrt{3} $
        \task $ \frac{1}{2} \sqrt{2} $
    \end{tasks}

\section*{Problem 2}

Credit: Sungmin Park

Suppose that the only consumption good in the state of Ohio is soybeans, and suppose that Ohio's output Y of number of soybeans is determined by the following production function

%
\begin{equation}
\label{eq:production}
    F( K, L ) = z K^{\frac{1}{3}} L^{\frac{2}{3}}
,\end{equation}
%
where $z$ is TFP, $K$ is its current stock of capital, and $L$ is the size of its current labor force.
Suppose also that the current technological level is $z=1$, the current stock of capital is 1,000, and the current labor force is 8,000.

(a) At the current technological level, capital stock, and labor force, what is the total output of soybeans? What is the marginal product of capital? What is the marginal product of labor?

\begin{center}
    $ MPK =$ \textbf{\red{\underline{\quad Q13 \quad}}}

    $ MPN =$ \textbf{\red{\underline{\quad Q14 \quad}}}
\end{center}

    \Question
    \answer{B}
    \begin{tasks}(4)
        \task $ 0.333 $
        \task $ 1.333 $
        \task $ 2.333 $
        \task $ 3.333 $
    \end{tasks}

    \Question
    \answer{A}
    \begin{tasks}(4)
        \task $ 0.333 $
        \task $ 1.333 $
        \task $ 2.333 $
        \task $ 3.333 $
    \end{tasks}


Suppose that the markets for capital and labor are competitive---that is, there are many firms (soybean farms) in Ohio, so that they are price-takers.

(b) What is the real rental rate of capital? What is the real wage? How many soybeans go to the capitalists? (i.e. what is the capital income?) How many soybeans go to the workers? (i.e. what is the labor income?) What is the total income?

\begin{center}
    rental rate $ R =  $ \textbf{\red{\underline{\quad Q15 \quad}}}

    wage $ w =  $ \textbf{\red{\underline{\quad Q16 \quad}}}

    Capital income $ R \times K $  \textbf{\red{\underline{\quad Q17 \quad}}}

    Labor income $ w \times L $  \textbf{\red{\underline{\quad Q18 \quad}}}

    Total income $ R \times K + w \times L $  \textbf{\red{\underline{\quad Q19 \quad}}}
\end{center}

    \Question
    \answer{D}
    \begin{tasks}(4)
        \task $ \frac{1}{3} $
        \task $ \frac{2}{3} $
        \task $ \frac{3}{3} $
        \task $ \frac{4}{3} $
    \end{tasks}

    \Question
    \answer{A}
    \begin{tasks}(4)
        \task $ \frac{1}{3} $
        \task $ \frac{2}{3} $
        \task $ \frac{3}{3} $
        \task $ \frac{4}{3} $
    \end{tasks}

    \Question
    \answer{B}
    \begin{tasks}(4)
        \task $ \frac{2000}{3} $
        \task $ \frac{4000}{3} $
        \task $ \frac{6000}{3} $
        \task $ \frac{8000}{3} $
    \end{tasks}

    \Question
    \answer{D}
    \begin{tasks}(4)
        \task $ \frac{2000}{3} $
        \task $ \frac{4000}{3} $
        \task $ \frac{6000}{3} $
        \task $ \frac{8000}{3} $
    \end{tasks}

    \Question
    \answer{B}
    \begin{tasks}(4)
        \task $ 2000 $
        \task $ 4000 $
        \task $ 6000 $
        \task $ 8000 $
    \end{tasks}


Now suppose that the demand for soybeans in Ohio are as follows. First, the demand for the consumption of soybeans is given as

%
\begin{equation}
\label{eq:demand}
    C = 0.5( Y - T ) + 500
,\end{equation}
%
where $T = 400$ is the number of soybeans taxed by the Ohio state government. Second, the demand for investing soybeans for future production is given as

%
\begin{equation}
\label{eq:investment}
    I = 1500 - 100r
,\end{equation}
%
where r is the real interest rate in percent. Finally, the demand for soybeans from the Ohio state government is  as the government collect the same amount as taxes T and spend it on running the Ohio State University.

(c) What are the equilibrium interest rate, investment, and consumption?

\begin{center}
    $ r =  $   \textbf{\red{\underline{\quad Q20 \quad}}}

    $ I =  $   \textbf{\red{\underline{\quad Q21 \quad}}}

    $ C =  $   \textbf{\red{\underline{\quad Q22 \quad}}}
\end{center}

    \Question
    \answer{A}
    \begin{tasks}(4)
        \task $ 2 $
        \task $ 3 $
        \task $ 4 $
        \task $ 5 $
    \end{tasks}

    \Question
    \answer{C}
    \begin{tasks}(4)
        \task $ 1500 $
        \task $ 1400 $
        \task $ 1300 $
        \task $ 1200 $
    \end{tasks}

    \Question
    \answer{C}
    \begin{tasks}(4)
        \task $ 2100 $
        \task $ 2200 $
        \task $ 2300 $
        \task $ 2400 $
    \end{tasks}


(d) Suppose the Ohio state government reduces the taxation to $ T = 200 $ while maintaining the same spending of $ G = 400 $, as a part of COVID-19 stimulus package.
What is the government surplus/deficit? What is the equilibrium interest rate, investment, and consumption after this government policy?

\begin{center}
    government surplus/deficit: $ T - G = $ \textbf{\red{\underline{\quad Q23 \quad}}}

    $ r =  $   \textbf{\red{\underline{\quad Q24 \quad}}}

    $ I =  $   \textbf{\red{\underline{\quad Q25 \quad}}}

    $ C =  $   \textbf{\red{\underline{\quad Q26 \quad}}}
\end{center}

    \Question
    \answer{C}
    \begin{tasks}(4)
        \task $ 200 $
        \task $ 400 $
        \task $ -200 $
        \task $ -400 $
    \end{tasks}


    \Question
    \answer{B}
    \begin{tasks}(4)
        \task $ 2 $
        \task $ 3 $
        \task $ 4 $
        \task $ 5 $
    \end{tasks}

    \Question
    \answer{D}
    \begin{tasks}(4)
        \task $ 1500 $
        \task $ 1400 $
        \task $ 1300 $
        \task $ 1200 $
    \end{tasks}

    \Question
    \answer{D}
    \begin{tasks}(4)
        \task $ 2100 $
        \task $ 2200 $
        \task $ 2300 $
        \task $ 2400 $
    \end{tasks}

Suppose instead that Ohio state residents' consumption demand also depends on the real interest rate:

%
\begin{equation}
\label{eq:demand_interestrate}
    C = 0.5 ( Y-T ) + 600 - 50r
.\end{equation}
%
That is, for every percentage point increase in the interest rate, the Ohio state resident would rather not consume 50 soybeans and save them instead.

(e) Analyze again the effects of reduced taxation to $ T = 200 $, using the new consumption demand but using everything else the same as (d)

\begin{center}
    government surplus/deficit: $ T - G = $ \textbf{\red{\underline{\quad Q27 \quad}}}

    $ r =  $   \textbf{\red{\underline{\quad Q28 \quad}}}

    $ I =  $   \textbf{\red{\underline{\quad Q29 \quad}}}

    $ C =  $   \textbf{\red{\underline{\quad Q30 \quad}}}
\end{center}

    \Question
    \answer{C}
    \begin{tasks}(4)
        \task $ 200 $
        \task $ 400 $
        \task $ -200 $
        \task $ -400 $
    \end{tasks}

    \Question
    \answer{B}
    \begin{tasks}(4)
        \task $ 1.667 $
        \task $ 2.667 $
        \task $ 1.333 $
        \task $ 2.333 $
    \end{tasks}

    \Question
    \answer{C}
    \begin{tasks}(4)
        \task $ 1537 $
        \task $ 3212 $
        \task $ 1233 $
        \task $ 2367 $
    \end{tasks}

    \Question
    \answer{D}
    \begin{tasks}(4)
        \task $ 1537 $
        \task $ 3212 $
        \task $ 1233 $
        \task $ 2367 $
    \end{tasks}

\end{Exercise}


\end{document}

