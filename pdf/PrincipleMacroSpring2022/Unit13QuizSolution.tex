\documentclass{beamer}
\usefonttheme[onlymath]{serif}
\usetheme[numbering=fraction, progressbar=frametitle]{metropolis}
\usepackage{appendixnumberbeamer}
\setbeamertemplate{navigation symbols}{}
%%% headline section guide
% \useoutertheme[footline=empty,subsection=false]{miniframes}
% \useinnertheme{circles}
%%% if wish no circles on headline
% \setbeamertemplate{headline}{%
%     \begin{beamercolorbox}[colsep=1.5pt]{upper separation line head}
%     \end{beamercolorbox}
%     \begin{beamercolorbox}{section in head/foot}
%     \vskip2pt\insertsectionnavigationhorizontal{\paperwidth}{}{}\vskip2pt
%         \end{beamercolorbox}%
%         \begin{beamercolorbox}[colsep=1.5pt]{lower separation line head}
%     \end{beamercolorbox}
% }

\usepackage[utf8]{inputenc}
\usepackage[T1]{fontenc}
\usepackage{textcomp}
\usepackage{amsmath, amssymb, amsthm}

\usepackage[style=authoryear,maxbibnames=9,maxcitenames=2,uniquelist=false,backend=biber,doi=false,url=false]{biblatex}
\renewcommand*{\nameyeardelim}{\addcomma\space} % have comma in parencite
\addbibresource{$BIB} % bibtex location
%%% Small bibliography slide
\setbeamertemplate{bibliography item}[triangle]
\makeatletter
\newcommand{\srcsize}{\@setfontsize{\srcsize}{6.5pt}{6.5pt}}
\makeatother
\renewcommand*{\bibfont}{\srcsize}

\usepackage{import}
\usepackage{pdfpages}
\usepackage{transparent}
\usepackage{xcolor}

\newcommand{\blue}[1]{\textcolor{blue}{#1}}
\newcommand{\red}[1]{\textcolor{red}{#1}}

\graphicspath{ {./figures} }
\newcommand{\inkfig}[2][1]{%
    \def\svgwidth{#1\columnwidth}
    \import{./figures/}{#2.pdf_tex}
}

%%%%%% Template
\usepackage{hyperref}
\definecolor{links}{HTML}{2A1B81}

%% beaver (red) style:
% \usecolortheme{beaver}
% \setbeamercolor{block body}{bg=gray!30!white}
% \setbeamercolor{block title}{bg=darkred!70, fg=black!2}
% \hypersetup{colorlinks=true,allcolors=red}

%% seahorse style:
\usecolortheme{seahorse}
\setbeamercolor{block body}{bg=mDarkTeal!30}
\setbeamercolor{block title}{bg=mDarkTeal,fg=black!2}
\hypersetup{colorlinks=true,allcolors=links}
%%%%%% Template

\pdfsuppresswarningpagegroup=1

\title{Unit 13 Quiz Solution}
\author{Hui-Jun Chen}
\institute{The Ohio State University}
\date{\today}

\begin{document}

\maketitle

% \frame{%
%    \maketitle
%    \begin{center}
%        \includegraphics
%			[width=0.2\textwidth]
%			{./figures/Ohio_State_University_seal}
%    \end{center}
% }


\begin{frame}{Q1}
    A new neighbour holds a party in order to get to know people. You notice that each time he greets someone he says his name is Joe and then says he’s a dentist. Which of the following is the most plausible explanation for why he tells people his occupation?
    \begin{itemize}
        \item It signals that if a neighbour has a dental problem, they know where to get treatment.
        \item He thinks it confers a superior status.
        \item \red{It’s a powerful but simple signal to neighbours about his income, education, tastes, attitudes and other characteristics.}

        \item He does not wish to be mistaken for a doctor or accountant.
    \end{itemize}

\end{frame}


\begin{frame}{Q2}
\label{slide:Q2}
    Your current income is about £3,000 per month. When a researcher asks you what level of compensation you would feel adequate in the case of job loss, you estimate this at £5,000 per month. This discrepancy definitely indicates:
    \begin{itemize}
        \item You think you are underpaid.
        \item You are greedy.
        \item You value the loss of status and general sense of wellbeing at £5,000 per month.
        \item \red{You value the loss of status and general sense of wellbeing at £2,000 per month.}
    \end{itemize}

\end{frame}

\begin{frame}{Q3}
\label{slide:Q3}
    The figure above shows the log of UK real GDP per capita between 1875 and 1914. Which of the following is correct? \href{https://drive.google.com/file/d/1LyYAV0U9japlKBuQQsD6nOercA-FlNkn/view?usp=drivesdk}{graph}
    \begin{itemize}
        \item \red{The growth of GDP in the 1950s was above the long-run average.}
        \item The growth of GDP in the 1880s was above the long-run average.
        \item In the figure the coefficient on x is 0.0156 and tells us the slope of the regression line (line of best fit). If, instead, this were 0.02 the line would be flatter.
        \item The slope of the blue line tells us the rate of growth of per capita GDP but can tell us nothing about the level.
    \end{itemize}



\end{frame}

\begin{frame}{Q4}
\label{slide:Q4}

    The two figures above both show the long-run growth of real per capita GDP in the UK (in logs) but for different periods. Which of the following is correct?
    \href{https://drive.google.com/file/d/1LyYAV0U9japlKBuQQsD6nOercA-FlNkn/view?usp=drivesdk}{graph 1}
    \href{https://drive.google.com/file/d/14ayMaL2Lfb2-APF5vq_Q96I9MbFk8fPX/view?usp=drivesdk}{graph 2}
    \begin{itemize}
        \item \red{Since 1950, the annual rate of growth of GDP per capita has been more rapid than the average rate of growth over the whole period since 1875.}
        \item The average rate of growth before 1950 must have been more rapid than since 1950.
        \item For the period 1875-1950 the average rate of growth in real GDP per capita must have been between 1.5 and 2.0 per cent.
        \item Since the financial crisis, the growth of real per capita GDP has fallen below the long-run average rate from 1875.
    \end{itemize}
\end{frame}

\begin{frame}{Q5}
\label{slide:Q5}
    Okun's law implies that:
    \begin{itemize}
        \item Changes in the unemployment rate and GDP growth rate are positively correlated.
        \item \red{The change in unemployment is negative in booms and positive in recessions.}
        \item Every change in the level of employment is exactly matched by an opposite change in the level of unemployment.
        \item GDP growth rate and changes in unemployment are inversely correlated.
    \end{itemize}
\end{frame}














\metroset{numbering=none}
\printbibliography[heading=none]
% \begin{frame}[allowframebreaks, noframenumbering]
%     \frametitle{References}
%     \printbibliography[heading=none]
% \end{frame}

\end{document}

