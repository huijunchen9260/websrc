\documentclass{beamer}
\usefonttheme[onlymath]{serif}
\usetheme[numbering=fraction, progressbar=frametitle]{metropolis}
\usepackage{appendixnumberbeamer}

\usepackage[utf8]{inputenc}
\usepackage[T1]{fontenc}
\usepackage{textcomp}
\usepackage{amsmath, amssymb, amsthm}

\usepackage[style=authoryear,maxbibnames=9,maxcitenames=2,uniquelist=false,backend=biber,doi=false,url=false]{biblatex}
\renewcommand*{\nameyeardelim}{\addcomma\space} % have comma in parencite
\addbibresource{$BIB} % bibtex location
%%% Small bibliography slide
\setbeamertemplate{bibliography item}[triangle]
\makeatletter
\newcommand{\srcsize}{\@setfontsize{\srcsize}{6.5pt}{6.5pt}}
\makeatother
\renewcommand*{\bibfont}{\srcsize}

\usepackage{import}
\usepackage{pdfpages}
\usepackage{transparent}
\usepackage{xcolor}

\newcommand{\blue}[1]{\textcolor{blue}{#1}}
\newcommand{\red}[1]{\textcolor{red}{#1}}

\graphicspath{ {./figures} }
\newcommand{\inkfig}[2][1]{%
    \def\svgwidth{#1\columnwidth}
    \import{./figures/}{#2.pdf_tex}
}

%%%%%% Template
\usepackage{hyperref}
\definecolor{links}{HTML}{2A1B81}

%% beaver (red) style:
% \usecolortheme{beaver}
% \setbeamercolor{block body}{bg=gray!30!white}
% \setbeamercolor{block title}{bg=darkred!70, fg=black!2}
% \hypersetup{colorlinks=true,allcolors=red}

%% seahorse style:
\usecolortheme{seahorse}
\setbeamercolor{block body}{bg=mDarkTeal!30}
\setbeamercolor{block title}{bg=mDarkTeal,fg=black!2}
\hypersetup{colorlinks=true,allcolors=links}
%%%%%% Template

\pdfsuppresswarningpagegroup=1

\title{Economics in my eyes}
\author{Hui-Jun Chen}
\institute{The Ohio State University}
\date{\today}

\begin{document}

\maketitle

\begin{frame}[standout]
    This is only plastic.

    It's just something made up by people.

    Truly meaningless until we put out faith in it.

    Faith is what makes an Economy exists.

    Without faith...it is only plastic cards and paper money.

    -- \href{https://youtu.be/YWva4HqVVw0}{South Park S13E3 Margaritaville}
\end{frame}

\begin{frame}{Are you...serious?!}
\label{slide:Are_you___serious__}
    Yes.

    Macroeconomics is a study on \textit{aggregate} behavior, and what the majority believes (not) to be valuable will eventually become (not) valuable.

    E.g. Value of Bitcoin in 2008 v.s. now.

    \href{https://en.wikipedia.org/wiki/Rational_expectations}{Rational Expectation}: agents inside the model are assumed to ``know the model'' and on average take the model's predictions as valid.
    Agents' some particular expectations may be wrong, but are correct \textit{on average} over time.
\end{frame}

\begin{frame}{The rational assumption is nonsense}
\label{slide:The_rational_assumption_is_nonsense}
    First of all, rational expectation $ \neq $ individual rationality.

    Your suspicion is probably valid in individual level.

    At aggregate level, \textbf{I believe} that the irrational behavior between each individual will average out, and eventualy the \textbf{aggregate behavior} can be rational.

    I wish I can prove this point using the great history of United State...
\end{frame}

\begin{frame}[standout]

    \href{https://youtu.be/QLq6GEiHqR8?t=520}{The Oregon Treaty (1846)}: 8:40-10:15

    \href{https://youtu.be/QLq6GEiHqR8?t=1016}{The Pig shooting}: 16:56-19:45

    \href{https://youtu.be/QLq6GEiHqR8?t=1213}{William S. Harney}: 20:22-23:08

    \href{https://youtu.be/QLq6GEiHqR8?t=1579}{Royal Navy Refuse Landing on San Juan Island}: 26:19-28:30

    \href{https://youtu.be/QLq6GEiHqR8?t=1794}{Winfield Scott \& Resolution}: 29:54-

    -- The Pig War (Video by \href{https://youtu.be/QLq6GEiHqR8}{Oversimplified})

\end{frame}


\end{document}
